\section{Example Section}

This is an example section; 
the following content is auto-generated dummy text.

\subsection{Example Sub-section}

% \begin{figure}[t]
% \centering
% foo
% \caption{This is an example figure.}
% \label{fig}
% \end{figure}

% \begin{table}[t]
% \centering
% \begin{tabular}{|cc|c|}
% \hline
% foo      & bar      & baz      \\
% \hline
% $0     $ & $0     $ & $0     $ \\
% $1     $ & $1     $ & $1     $ \\
% $\vdots$ & $\vdots$ & $\vdots$ \\
% $9     $ & $9     $ & $9     $ \\
% \hline
% \end{tabular}
% \caption{This is an example table.}
% \label{tab}
% \end{table}

% \begin{algorithm}[t]
% \For{$i=0$ {\bf upto} $n$}{
%   $t_i \leftarrow 0$\;
% }
% \caption{This is an example algorithm.}
% \label{alg}
% \end{algorithm}

% \begin{lstlisting}[float={t},caption={This is an example listing.},label={lst},language=C]
% for( i = 0; i < n; i++ ) {
%   t[ i ] = 0;
% }
% \end{lstlisting}

\begin{mathpar}
  \inferH{Axiom}{
  }{
    \IsProc{\Link{x}{y}}{\Name{x} : \negg{A}, \Name{y} : A}
  }\qquad
  \inferH{Tensor}{
      \IsProc{P}{\Gamma, \Name{x} : A} \\
      \IsProc{Q}{\Delta, \Name{y} : B}
  }{
      \IsProc{\Out{y}{x}{(P|Q)}}{\Gamma, \Delta, \Name{y} : A \otimes B}
  } \qquad
  \inferH{Par}{
        \IsProc{P}{\Gamma, \Name{x} : A, \Name{y} : B} 
  }{
        \IsProc{\In{y}{x}{P}}{\Gamma, \Name{y} : A \parr B}
  }
\end{mathpar}

\begin{mathpar}
  \inferH{Plus-L}{
    \IsProc{P}{\Gamma, \Name{x} : A}
  }{
    \IsProc{\Inl{x}{P}}{\Gamma, \Name{x} : A \oplus B}
  } \qquad
  \inferH{Plus-R}{
    \IsProc{P}{\Gamma, \Name{x} : B}
  }{
    \IsProc{\Inr{x}{P}}{\Gamma, \Name{x} : A \oplus B}
  }
   \qquad
  \inferH{With}{
    \IsProc{P}{\Gamma, \Name{x} : A} \\
    \IsProc{Q}{\Gamma, \Name{x} : B}
  }{
    \IsProc{\Casep{x}{P}{Q}}{\Gamma, \Name{x} : A \with B}
  }
  \end{mathpar}

\begin{mathpar}
  \inferH{Cut}{
    \IsProc{P}{\Gamma, \Name{x} : \negg{A}}\\
    \IsProc{Q}{\Gamma, \Name{x} : {A}}   
  }{
    \IsProc{\New*{x}{}{P|Q}}{\Gamma, \Delta}
  }\qquad
  \inferH{Weakening}{
    \IsProc{P}{\Gamma}
  }{
    \IsProc{P}{\Gamma, \Name{x} : \whynot A}
  }\qquad
  \inferH{Contraction}{
    \IsProc{P}{\Gamma, \Name{x} : \whynot A, \Name{y} : \whynot A}
  }{
    \IsProc{P \Out{}{x/y}{}}{\Gamma, \Name{x} : \whynot A}
  }\qquad
  \inferH{Dereliction}{
    \IsProc{P}{\Gamma, \Name{x} : A}
  }{
    \IsProc{\Out{\whynot y}{x}{P}}{\Gamma, \Name{y} : \whynot A}
  }\qquad
  \inferH{Promotion}{
    \IsProc{P}{\whynot \Gamma, \Name{x} : A}
  }{
    \IsProc{\In{\ofc y}{x}{P}}{\whynot \Gamma, \Name{y} : \ofc A}
  }
\end{mathpar}

\begin{mathpar}
  \inferH{}{
        \inferH{}{
              \IsProc{P}{\Gamma, \Name{x} : \negg{A}} \\
              \IsProc{Q}{\Delta, \Name{y} : \negg{B}}
        }{
        \IsProc{\Out{y}{x}{(P|Q)}}{\Gamma, \Delta, \Name{y} : \negg{A} \otimes \negg{B}} \\
  }\\
        \inferH{}{
               \IsProc{R}{\Theta, \Name{x} : A, \Name{y} : B}
        }{
               \IsProc{\In{y}{x}{R}}{\Theta, \Name{y} : A \parr B}
        }
      }{
        \IsProc{\New*{y}{}{\Out{y}{x}{(P|Q)}|\In{y}{x}{R}}}{\Gamma, \Delta, \Theta}
  }\\
  \inferH{}{
        \inferH{}{
              \IsProc{P}{\Gamma, \Name{x} : \negg{A}} \\
              \IsProc{R}{\Theta, \Name{x} : A, \Name{y}: B}
        }{
        \IsProc{\New*{x}{}{P|Q}}{\Gamma, \Theta, \Name{y} : B} \\
  }\\
        \inferH{}{
        }{
               \IsProc{Q}{\Delta, \Name{y} : \negg{B}}
        }
      }{
        \IsProc{\New*{y}{}{\New*{x}{}{P|R}|Q}}{\Gamma, \Delta, \Theta}
  }
\end{mathpar}

The \ruleref{Axiom} states that if we have some variable A, then we also have its dual $\negg{A}$.
The \ruleref{Tensor} rule outputs a fresh channel x along y, then continues as P and Q in parallel.
The rule \ruleref{Par} inputs A, then continues as B. 

The \ruleref{Plus-L} indicates left selection, and similarly \ruleref{Plus-L} indicates right selection.
The \ruleref{With} rule offers a choice between processes \Proc{P} and \Proc{Q}.

% This is an example sub-section;
% the following content is auto-generated dummy text.
% Notice the examples in Figure~\ref{fig}, Table~\ref{tab}, Algorithm~\ref{alg}
% and Listing~\ref{lst}.

\subsubsection{Example Sub-sub-section}

This is an example sub-sub-section;
the following content is auto-generated dummy text.

\paragraph{Example paragraph.}

This is an example paragraph; note the trailing full-stop in the title,
which is intended to ensure it does not run into the text.

% -----------------------------------------------------------------------------

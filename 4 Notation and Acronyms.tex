\chapter*{Notation and Acronyms}

{\bf An optional section, of roughly $1$ or $2$ pages}
\vspace{1cm} 

\noindent
Any well written document will introduce notation and acronyms before
their use, {\em even if} they are standard in some way: this ensures 
any reader can understand the resulting self-contained content.  

Said introduction can exist within the dissertation itself, wherever 
that is appropriate.  For an acronym, this is typically achieved at 
the first point of use via ``Advanced Encryption Standard (AES)'' or 
similar, noting the capitalisation of relevant letters.  However, it 
can be useful to include an additional, dedicated list at the start 
of the dissertation; the advantage of doing so is that you cannot 
mistakenly use an acronym before defining it.  A limited example is 
as follows:

\begin{quote}
\noindent
\begin{tabular}{lcl}
AES                 &:     & Advanced Encryption Standard                                         \\
DES                 &:     & Data Encryption Standard                                             \\
                    &\vdots&                                                                      \\
${\mathcal H}( x )$ &:     & the Hamming weight of $x$                                            \\
${\mathbb  F}_q$    &:     & a finite field with $q$ elements                                     \\
$x_i$               &:     & the $i$-th bit of some binary sequence $x$, st. $x_i \in \{ 0, 1 \}$ \\
\end{tabular}
\end{quote}

% -----------------------------------------------------------------------------

\chapter*{Summary of Changes}

{\bf A conditional section, of at most $1$ page} 
\vspace{1cm} 

Iff. the dissertation represents a resubmission (e.g., as the result of
a resit), this section is compulsory: the content should summarise all
non-trivial changes made to the initial submission.  Otherwise you can
omit it, since a summary of this type is clearly nonsensical.

When included, the section will ideally be used to highlight additional
work completed, and address criticism raised in any associated feedback.
Clearly it is difficult to give generic advice about how to do so, but
an example might be as follows:

\begin{quote}
\noindent
\begin{itemize}
\item Feedback from the initial submission criticised the design and 
      implementation of my genetic algorithm, stating ``there seems 
      to have been no attention to computational complexity during the
      design, and obvious methods of optimisation are missing within
      the resulting implementation''.  Chapter $3$ now includes a
      comprehensive analysis of the algorithm, in terms of both time
      and space.  While I have not altered the algorithm itself, I
      have included a cache mechanism (also detailed in Chapter $3$)
      that provides a significant improvement in average run-time.
\item I added a feature in my implementation to allow automatic rather
      than manual selection of various parameters; the experimental
      results in Chapter $4$ have been updated to reflect this.
\item Questions after the presentation highlighted a range of related
      work that I had not considered: I have make a number of updates 
      to Chapter $2$, resolving this issue.
\end{itemize}
\end{quote}

% -----------------------------------------------------------------------------

\chapter{Classical Linear Logic}
\label{chap:execution}

Classical linear logic (CLL) is a sequent calculus which differs from classical logic in a few ways.
The first is the addition of linear negation.
In classical linear logic variables are denoted by capital letters $A, B, ...$ and the duals to these 
variables are denoted $\negg{A}, \negg{B}, ...$ where $\negg{A}$ is the negation of A such that 
$\negg{\negg{A}} = A$. This negation is not present in intuitionistic linear logic, but is a large part 
of classical linear logic and all propositions have a dual. 

The next difference is that we replace the usual two-sided sequent with a one-sided sequent, so a sequent 
in classical logic such as:
\begin{mathpar}
  A_1, A_2, ...,A_n \vdash B_1, B_2, ..., B_m
\end{mathpar}
would read as
\begin{mathpar}
  \vdash \negg{A_1}, \negg{A_2}, ..., \negg{A_n}, B_1, B_1, ..., B_m
\end{mathpar}
in classica linear logic.
This makes CLL slightly easier to read as we don't have to worry about propositions on the left-hand 
side of our sequent. As we can see from our new one-sided sequent, we now also have 

We also have that any proposition must be used exactly as many times as it appears. We cannot have some 
proposition A and not make use of it, and similarly we cannot use it any more than once if it only appears 
once. 

Our propositions for CLL are as follows:

\begin{figure}[h]
  \begin{align*}
      & X &\text{variable} \\
      & \negg{X} &\text{variable dual} \\
      & A \otimes B &\text{tensor} \\
      & A \parr B &\text{par} \\
      & A \oplus B &\text{plus} \\
      & A \with B &\text{with} \\
      & \ofc A &\text{of course} \\
      & \whynot A &\text{why not} \\
      & 1 &\text{tensor unit} \\
      & \bot &\text{par unit} \\
      & 0 &\text{plus unit} \\
      & \top &\text{with unit} \\
  \end{align*}
  \caption{Propositions for Classical Linear Logic}
  \label{fig: p cll}
\end{figure}

\noindent
Generally when researching the basics of classical linear logic, one may find the vending machine example. It is 
widely used to describe the propositions of CLL in a way that is easy to understand. Say we have a vending machine 
which takes \pounds 1 coins and has the options of Tea or Coffee. If we see the tensor symbol on the machine: \emph{Tea} 
$\otimes$ \emph{Coffee}, then inserting \pounds 1 will give us both Tea and Coffee. However if we see \emph{Tea} $\with$ \emph{Coffee} then 
inserting \pounds 1 will give us a choice of either Tea or Coffee. If you can't decide what drink you want then finding a 
vending machine which says \emph{Tea} $\oplus$ \emph{Coffee} will take our \pounds 1 and give us either Tea or Coffee making the decision 
for us. Par is not quite so easy to describe, if we stay with our vending machines then we would most likely see 
$\pounds 1 \ \parr$ \emph{Tea} which would imply that inserting \pounds 1 into the vending machine will then give us Tea. 
Having a term $A \parr B$ in classical linear logic is equivalent to having a term $\negg{A} \multimap B$ in linear logic. 
The exponential fragment also fits in with this slightly differently. If we have a term $\ofc \pounds 1$ this would indicate 
that we have multiple \pounds 1 coins which we may use at the vending machine. We may also see a term $\whynot \pounds 1$ 
on any of these vending machines which would indicate that it is possible to use the machine more than once by inserting 
multiple \pounds 1 coins. Both of course ($\ofc$) and why not ($\whynot$) mean that whatever follows it can be used any 
number of times, or even never used at all.
\\\\
\noindent
For this report, we will modify these definitions slightly so they fit with our process calculus.

Tensor, $A \otimes B$ represents multiplicative conjunction and means output A, then continue as B. The dual to tensor 
is par which represents multiplicative disjunction. We would read $A \parr B$ as input A, then continue as B. 

We have plus for additive disjunction so $A \oplus B$ means select either A or B.  Dual to this we have with: 
$A \with B$ meaning offer a choice between A and B. 

Our exponential component consists of of course and why not, where 
$\ofc A$ means we have a server which can accept many copies of A, and $\whynot A$ means we have a client 
who may request many copies of A. 

It is important to note that every proposition has a dual, specifically:

\begin{align*}
  \negg{(A \otimes B)} &= \negg{A} \parr \negg{B} \\
  \negg{(A \oplus B)} &= \negg{A} \with \negg{B} \\
  \negg{(\ofc A)} &= \whynot \negg{A}
\end{align*}

\noindent
Classical linear logic naturally lends itself to parallelism, so we present the rules of CLL 
alongside a process calculus similar to $\pi$ calculus. The processes for this are as follows:

\begin{figure}[h]
  \begin{align*}
    \Proc{P} ::= & \\
    & \Proc{P \mathbin{|} Q} & \text{parallel composition} \\
    & \Proc{\Link{x}{y}} & \text{link \Name{x} with \Name{y}} \\
    & \Proc{\Out{y}{x}{P}} & \text{output \Name{x} on channel \Name{y}} \\
    & \Proc{\In{y}{x}{P}} & \text{input \Name{x} on channel \Name{y}} \\
    & \Proc{\Inl{x}{P}} & \text{left selection} \\
    & \Proc{\Inr{x}{P}} & \text{right selection} \\
    & \Proc{\Casep{x}{P}{Q}} & \text{choice} \\
    & \Proc{\New*{x}{}{P \mathbin{|} Q}} & \text{connect on channel \Name{x}} \\
    & \Proc{\Out{\whynot y}{x}{P}} & \text{client request} \\
    & \Proc{\In{\ofc y}{x}{P}} & \text{server accept} \\
    & \Proc{\Out{y}{\ }{P}} & \text{empty output} \\
    & \Proc{\In{y}{\ }{P}} & \text{empty input} \\
    & \Proc{\Casep{x}{}{}} & \text{empty choice}
  \end{align*}
  \caption{Classical Processes (CP)}
  \label{fig: p cp}
\end{figure}

\noindent
A process $\Proc{P}$ in CP can take on a few forms. Link says that we can connect two channels in 
such a way that any message delivered to \Name{y} will be forwarded over \Name{x} and vice verca.
For an output such as $\Proc{\Out{y}{x}{P \mathbin{|} Q}}$ we have that $\Name{y}$ is bound in 
$\Proc{P}$, but is not bound in $\Proc{Q}$. For the input $\Proc{\In{y}{x}{P}}$, $\Name{y}$ is bound 
in $\Proc{P}$. In $\Proc{\New*{x}{}{P \mathbin{|} Q}}$, $\Name{x}$ is bound in both $\Proc{P}$ and 
$\Proc{Q}$. Server accept is just like input so for $\Proc{\In{\ofc y}{x}{P}}$, we would have that 
$\Name{x}$ is bound in $\Proc{P}$, and similarly for $\Proc{\Out{\whynot y}{x}{P}}$. 

\section{Rules for CLL}

\noindent
We can now combine our process calculus CP with our propositions for CLL and present the rules for 
classical linear logic alongside their corresponding processes. Our typing judgements have the form 
$\IsProc{P}{\Gamma, \Name{x}: A}$ where \Proc{P} is a CP process, $\Gamma$ is a type environment, 
\Name{x} is a channel, and $A$ is a CLL proposition. This typing judgement means we have some process 
\Proc{P} communicating along channel \Name{x} obeying protocol $A$. Processes can communicate over 
multiple channels following different protocols, and as we saw in \refname{fig: p cp} the processes 
may also take on varying forms.

\begin{figure}[h]
  \begin{mathpar}
    \inferH{Axiom}{
    }{
      \IsProc{\Link{x}{y}}{\Name{x} : \negg{A}, \Name{y} : A}
    }\qquad
    \inferH{Cut}{
      \IsProc{P}{\Gamma, \Name{x} : \negg{A}}\\
      \IsProc{Q}{\Gamma, \Name{x} : {A}}   
    }{
      \IsProc{\New*{x}{}{P \mathbin{|} Q}}{\Gamma, \Delta}
    }\\

    \inferH{Tensor}{
        \IsProc{P}{\Gamma, \Name{x} : A} \\
        \IsProc{Q}{\Delta, \Name{y} : B}
    }{
        \IsProc{\Out{y}{x}{(P \mathbin{|} Q)}}{\Gamma, \Delta, \Name{y} : A \otimes B}
    } \qquad
    \inferH{Par}{
          \IsProc{P}{\Gamma, \Name{x} : A, \Name{y} : B} 
    }{
          \IsProc{\In{y}{x}{P}}{\Gamma, \Name{y} : A \parr B}
    }\\

    \inferH{Plus-L}{
      \IsProc{P}{\Gamma, \Name{x} : A}
    }{
      \IsProc{\Inl{x}{P}}{\Gamma, \Name{x} : A \oplus B}
    } \qquad
    \inferH{Plus-R}{
      \IsProc{P}{\Gamma, \Name{x} : B}
    }{
      \IsProc{\Inr{x}{P}}{\Gamma, \Name{x} : A \oplus B}
    }
    \qquad
    \inferH{With}{
      \IsProc{P}{\Gamma, \Name{x} : A} \\
      \IsProc{Q}{\Gamma, \Name{x} : B}
    }{
      \IsProc{\Casep{x}{P}{Q}}{\Gamma, \Name{x} : A \with B}
    }\\

    \inferH{Weakening}{
      \IsProc{P}{\Gamma}
    }{
      \IsProc{P}{\Gamma, \Name{x} : \whynot A}
    }\qquad
    \inferH{Contraction}{
      \IsProc{P}{\Gamma, \Name{x} : \whynot A, \Name{y} : \whynot A}
    }{
      \IsProc{P \Out{}{x/y}{}}{\Gamma, \Name{x} : \whynot A}
    }\qquad
    \inferH{Dereliction}{
      \IsProc{P}{\Gamma, \Name{x} : A}
    }{
      \IsProc{\Out{\whynot y}{x}{P}}{\Gamma, \Name{y} : \whynot A}
    }\\

    \inferH{Promotion}{
      \IsProc{P}{\whynot \Gamma, \Name{x} : A}
    }{
      \IsProc{\In{\ofc y}{x}{P}}{\whynot \Gamma, \Name{y} : \ofc A}
    }\\

    \inferH{}{
    }{
      \IsProc{\Out{x}{\ }{P}}{\Name{x}: 1}
    }\qquad
    \inferH{}{
      \IsProc{P}{\Gamma}
    }{
      \IsProc{\In{x}{\ }{P}}{\Gamma, \Name{x}: \bot}
    }\qquad
    \inferH{}{
    }{
      \IsProc{\Casep{x}{}{}}{\Gamma, \Name{x}: \top}
    }
  \end{mathpar}
  \caption{Rules for Classical Linear Logic with CP}
  \label{fig: r cll cp}
\end{figure}

\noindent
The \ruleref{Axiom} states that if we have some variable A, then we also have its dual $\negg{A}$. 
For the CP side of things, we see that if we have two channels following dual protocols then any 
input along \Name{x} is sent as output along \Name{y}.
The \ruleref{Cut} rule allows us to connect two processes together. As the protocols are dual to, 
any transmissions and selections over one correspond with receives and choices over the other. This 
along with communication only via one channel ensures that the processes cannot get stuck.

The \ruleref{Tensor} rule outputs a fresh channel x along y, then continues as P and Q in parallel.
As P and Q communicate over different channels, we have disjoint concurrency so the processes cannot 
communicate with each other. The new process, P|Q communicates over channel \Name{y}.
The rule \ruleref{Par} inputs A, then continues as B. This is known as connected concurrency as P can 
communicate along both \Name{x} and \Name{y}.

The rule \ruleref{Plus-L} indicates left selection, and similarly \ruleref{Plus-R} indicates right 
selection. The process $\Proc{\Inl{x}{P}}$ obeys protocol $A \oplus B$ by requesting the left option 
from a choice sent along $x$. The process for $\Proc{\Inr{x}{P}}$ is symmetric. 
The \ruleref{With} rule offers a choice between processes \Proc{P} and \Proc{Q}. The new process 
$\Proc{\Casep{x}{P}{Q}}$ will receive a selection over channel $x$ and execute either \Proc{P} or 
\Proc{Q} accordingly.

\ruleref{Weakening} lets us consider a process which doesn't communicate or follow a protocol to be 
a process which communicates along a channel $\Name{x}$ with protocol \whynot A. 
If a process \Proc{P} communicates along two channels, both following the same protocol \whynot A, 
then we can use \ruleref{Contraction} to substitute one channel for another so \Proc{P} communicates 
along only one channel following protocol \whynot A.
\ruleref{Dereliction} allows us to.

\section{Cut Reduction}

\noindent 
Just like in the simply typed lambda calculus, there are dynamic rules for classical linear logic. 
We have already seen the \ruleref{Cut} rule and the dynamics for CLL make use of this rule to simplify 
terms via cut reduction. 

\begin{mathpar}

  \inferH{}{
    \inferH{}{
    }{
      \IsProc{\Link{x}{y}}{\Name{x}: \negg{A}, \Name{y}: A}
    }\\
    \inferH{}{
    }{
      \IsProc{P}{\Gamma, \Name{x}: A}
    }
  }{
    \IsProc{\New*{x}{}{\Link{x}{y} \mathbin{|} P}}{\Gamma, \Name{y}: A}
  } \quad \Longrightarrow \quad
  \IsProc{\Out{P}{y/x}{}}{\Gamma, \Name{y}: A} \\\\

  \inferH{}{
    \inferH{}{
      \IsProc{P}{\Gamma, \Name{x} : \negg{A}} \\
      \IsProc{Q}{\Delta, \Name{y} : \negg{B}}
    }{
      \IsProc{\Out{y}{x}{(P \mathbin{|} Q)}}{\Gamma, \Delta, \Name{y} : \negg{A} \otimes \negg{B}} \\
    }\\
    \inferH{}{
      \IsProc{R}{\Theta, \Name{x} : A, \Name{y} : B}
    }{
      \IsProc{\In{y}{x}{R}}{\Theta, \Name{y} : A \parr B}
    }
  }{
    \IsProc{\New*{y}{}{\Out{y}{x}{(P \mathbin{|} Q)} \mathbin{|} \In{y}{x}{R}}}{\Gamma, \Delta, \Theta}
  } \quad \Longrightarrow \\

  \inferH{}{
    \inferH{}{
      \IsProc{P}{\Gamma, \Name{x} : \negg{A}} \\
      \IsProc{R}{\Theta, \Name{x} : A, \Name{y}: B}
    }{
      \IsProc{\New*{x}{}{P \mathbin{|} Q}}{\Gamma, \Theta, \Name{y} : B} \\
    }\\
    \inferH{}{
    }{
      \IsProc{Q}{\Delta, \Name{y} : \negg{B}}
    }
  }{
    \IsProc{\New*{y}{}{\New*{x}{}{P \mathbin{|} R} \mathbin{|} Q}}{\Gamma, \Delta, \Theta}
  }\\\\

  \inferH{}{
    \inferH{}{
      \IsProc{P}{\Gamma, \Name{x}: A}
    }{
      \IsProc{\Inl{x}{P}}{\Gamma, \Name{x}: A \oplus B}
    }\\
    \inferH{}{
      \IsProc{Q}{\Delta, \Name{x}: \negg{A}}\\
      \IsProc{R}{\Theta, \Name{x}: \negg{B}}
    }{
      \IsProc{\Casep{x}{Q}{R}}{\Delta, \Name{x}: A \with B}
    }
  }{
    \IsProc{\New*{x}{}{\Inl{x}{P} \mathbin{|} \Casep{x}{Q}{R}}}{\Gamma, \Delta}
  } \quad \Longrightarrow \quad
  \inferH{}{
    \IsProc{P}{\Gamma, \Name{x}: A}\\
    \IsProc{Q}{\Delta, \Name{x}: \negg{A}}
  }{
    \IsProc{\New*{x}{}{P \mathbin{|} Q}}{\Gamma, \Delta}
  }\\\\

  \inferH{}{
    \inferH{}{
      \IsProc{P}{\whynot \Gamma, \Name{x}: A}
    }{
      \IsProc{\In{\ofc y}{x}{P}}{\whynot \Gamma, \Name{y}: \ofc A}
    }\\
    \inferH{}{
      \IsProc{Q}{\Delta, \Name{x}: \negg{A}}
    }{
      \IsProc{\Out{\whynot y}{x}{Q}}{\Delta, \Name{y}: \negg{\whynot A}}
    }
  }{
    \IsProc{\New*{y}{}{\In{\ofc y}{x}{P} \mathbin{|} \whynot y}{x}{Q}}{\whynot \Gamma, \Delta}
  } \quad \Longrightarrow \quad
  \inferH{}{
    \IsProc{P}{\whynot \Gamma, \Name{x}: A}\\
    \IsProc{Q}{\Delta, \Name{x}: \negg{A}}
  }{
    \IsProc{\New*{x}{}{P \mathbin{|} Q}}{\whynot \Gamma, \Delta}
  }\\\\

  \inferH{}{
    \inferH{}{
      \IsProc{P}{\whynot \Gamma, \Name{x}: A}
    }{
      \IsProc{\In{\ofc y}{x}{P}}{\whynot \Gamma, \Name{y}: \ofc A}
    }\\
    \inferH{}{
      \IsProc{Q}{\Delta}
    }{
      \IsProc{Q}{\Delta, \Name{x}: \negg{\whynot A}}
    }
  }{
    \IsProc{\New*{y}{}{\In{\ofc y}{x}{P} \mathbin{|} Q}}{\whynot \Gamma, \Delta}
  } \quad \Longrightarrow \quad
  \inferH{}{
    \IsProc{Q}{\Delta}
  }{
    \IsProc{Q}{\whynot \Gamma, \Delta}
  }\\\\

  \inferH{}{
    \inferH{}{
      \IsProc{P}{\whynot \Gamma, \Name{x}: A}
    }{
      \IsProc{\In{\ofc y}{x}{P}}{\whynot \Gamma, \Name{y}: \ofc A}
    }\\
    \inferH{}{
      \IsProc{Q}{\Delta, \Name{y}: \whynot A, \Name{z}: \whynot A}
    }{
      \IsProc{Q[y/z]}{\Delta, \Name{y}: \whynot A}
    }
  }{
    \IsProc{\New*{y}{}{\In{\ofc y}{x}{P} \mathbin{|} Q[y/z]}}{\whynot \Gamma, \Delta}
  }\quad \Longrightarrow \\

  \inferH{}{
    \inferH{}{
      \IsProc{P}{\whynot \Gamma, \Name{x}: A}
    }{
      \IsProc{\In{\ofc y}{x}{P}}{\whynot \Gamma, \Name{y}: \ofc A}
    }\\
    \inferH{}{
      \inferH{}{
        \IsProc{P'}{\whynot \Gamma', \Name{w}: A}
      }{
        \IsProc{\In{\ofc z}{w}{P}}{\whynot \Gamma', \Name{z}: \ofc A}
      }\\
      \IsProc{Q}{\Delta, \Name{y}: \negg{\whynot A}, \Name{z}: \negg{\whynot A}}
    }{
      \IsProc{\New*{z}{}{\In{\ofc z}{w}{P} \mathbin{|} Q}}{\whynot \Gamma', \Delta, \Name{x}: \negg{\whynot A}}
    }
  }{
    \inferH{}{
      \IsProc{\New*{y}{}{\In{\ofc y}{x}{P} \mathbin{|} \New*{z}{}{\In{\ofc z}{w}{P'} \mathbin{|} Q}}}{\whynot \Gamma, \whynot \Gamma', \Delta}
    }{
      \IsProc{\New*{y}{}{\In{\ofc y}{x}{P} \mathbin{|} \New*{z}{}{\In{\ofc z}{y}{P} \mathbin{|} Q}}}{\whynot \Gamma, \Delta}
    }
  }\\\\

  \inferH{}{
    \inferH{}{
      \IsProc{P}{\Gamma}
    }{
      \IsProc{\In{x}{\ }{P}}{\Gamma, \Name{x}: \bot}
    }\\
    \inferH{}{
    }{
      \IsProc{\Out{x}{\ }{Q}}{\Name{x}: 1}
    }
  }{
    \IsProc{\New*{x}{}{}\In{x}{\ }{P} \mathbin{|} \Out{x}{\ }{Q}}{\Gamma}
  }\quad \Longrightarrow \quad
  \IsProc{P}{\Gamma}
\end{mathpar}





% {\bf A topic-specific chapter, of roughly $15$ pages} 
% \vspace{1cm} 

% \noindent
% This chapter is intended to describe what you did: the goal is to explain
% the main activity or activities, of any type, which constituted your work 
% during the project.  The content is highly topic-specific, but for many 
% projects it will make sense to split the chapter into two sections: one 
% will discuss the design of something (e.g., some hardware or software, or 
% an algorithm, or experiment), including any rationale or decisions made, 
% and the other will discuss how this design was realised via some form of 
% implementation.  

% This is, of course, far from ideal for {\em many} project topics.  Some
% situations which clearly require a different approach include:

% \begin{itemize}
% \item In a project where asymptotic analysis of some algorithm is the goal,
%       there is no real ``design and implementation'' in a traditional sense
%       even though the activity of analysis is clearly within the remit of
%       this chapter.
% \item In a project where analysis of some results is as major, or a more
%       major goal than the implementation that produced them, it might be
%       sensible to merge this chapter with the next one: the main activity 
%       is such that discussion of the results cannot be viewed separately.
% \end{itemize}

% \noindent
% Note that it is common to include evidence of ``best practice'' project 
% management (e.g., use of version control, choice of programming language 
% and so on).  Rather than simply a rote list, make sure any such content 
% is useful and/or informative in some way: for example, if there was a 
% decision to be made then explain the trade-offs and implications 
% involved.

% -----------------------------------------------------------------------------

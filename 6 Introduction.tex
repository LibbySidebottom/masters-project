\chapter{Introduction}
\label{chap: intro}

% The problem - no one ever wrote any basic explanation on CLL so the literature on the subject is very 
% inaccessible. The goal - to introduce CLL in an accessible format and to provide a translation from the 
% Lambda Calculus to CLL.

\noindent
There is no lack of literature on classical linear logic (CLL), however almost all of it assumes a reasonable 
level of prior knowledge on the subject. This would not be a problem if someone had written down all the 
rules and explained the purpose of CLL. Unfortunately, no one thought to do such a thing and so the rules 
are taught to students by supervisors in a one-on-one setting. The first aim of this report is to present 
the rules of classical linear logic such that someone who has no prior knowledge can come away knowing 
enough to approach more literature on the topic without feeling out of depth. We will then provide a 
translation from the simply typed lambda calculus into classical linear logic which has been considered 
by many prominent figures, but never been written out in full. \\

\noindent
There are two forms of logic: classical logic, and constructive logic (also called intuitionistic logic). 
Classical logic is the foundation for classical mathematics and has a few defining features including 
logical operations, structural rules, and double negation. Logical operations connect two or more statements 
such as $+, \otimes$, etc. Structural rules are operations performed directly on judgements or sequents such 
as weakening and exchange. Double negation is the assertation that if a proposition P is true then the inverse 
of it's inverse is also true, $P = \neg \neg P$. Statements in classical logic can be either 
true or false, and cannot be undetermined. More specifically, if we have some proposition P, either P is 
true, or the negation of P is true. This means we can use proof by contradiction to show that a proposition 
is true if its negation is false. Put simply, in classical logic, a statement is true if it is not false. \\

\noindent
In constructive logic, we have a stronger notion of truth. We only accept a proposition as true if it has a 
constructive proof, so in order to prove an object exists we must either provide an example of such an object or 
explain how we would create it. Dually, we only accept a proposition to be false if we have a refutation of it in 
constructive logic. There is no expectation that a proposition be either true or false, it can simply be a proposition. 
Constructive logic is a logic of positive evidence which makes it more expressive than classical logic. Any 
proof of a proposition in classical logic can be translated into a proof of a weaker (although calssically 
equivalent) proposition in constructive logic. \\

\noindent
Our main topic is classical linear logic which was first presented by Girard as an extension of constructive logic 
with a classical framework. CLL is a substructural logic which makes use of double negation from classical logic, and 
constructivism from intuitionistic logic. The addition of exponentials allows us to use structural rules to some extent. 
It has a number of applications in computer science, namely in reasoning about resources, resource useage, ownership, 
and communication. The latter is seen under the Curry-Howard isomorphism and is the aspect we will be focusing on. \\

\noindent
The Curry-Howard isomorphism is a correspondence between proof systems and type systems where we see propositions as 
types, proofs as programs, and normalisation as computation. Here we use a variant of this found by Caires and Pfenning 
\cite{caires2010} with 
\begin{align*}
    \text{propositions }& \text{as session types,} \\
    \text{proofs } & \text{as processes,} \\
    \text{cut elimination } & \text{as computation.} 
\end{align*}

\noindent
This variant equates classical linear logic with a process calculus similar to the $\pi$-calculus and allows 
us to model classical linear logic as a parallel programming language. For this report we will use Wadler's 
\cite{wadler2014} variant on the $\pi$-calculus which he combines with CLL to create a session-typed process calculus, 
classical processes (CP). Wadler's variant extends the $\pi$-calculus so it has processes corresponding to 
each of the typing judgements needed for CLL. For each logical proposition we have a session type which 
describes a type for the communicating protocols. Session types  allow us to describe how two processes 
communicate along a channel. They are most often used in relation to the $\pi$-calculus which was introduced 
by Milner \cite{milner1992} as a way to model processes which communicate concurrently. \\

\noindent
When Girard first introduced classical linear logic \cite{girard1987}, he suggested that the multiplicative fragment 
on CLL has obvious links to parallel computation. 

% -----------------------------------------------------------------------------
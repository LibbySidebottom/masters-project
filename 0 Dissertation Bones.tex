% The document class supplies options to control rendering of some standard
% features in the result.  The goal is for uniform style, so some attention 
% to detail is *vital* with all fields.  Each field (i.e., text inside the
% curly braces below, so the MEng text inside {MEng} for instance) should 
% take into account the following:
%
% - author name       should be formatted as "FirstName LastName"
%   (not "Initial LastName" for example),
% - supervisor name   should be formatted as "Title FirstName LastName"
%   (where Title is "Dr." or "Prof." for example),
% - degree programme  should be "BSc", "MEng", "MSci", "MSc" or "PhD",
% - dissertation title should be correctly capitalised (plus you can have
%   an optional sub-title if appropriate, or leave this field blank),
% - dissertation type should be formatted as one of the following:
%   * for the MEng degree programme either "enterprise" or "research" to
%     reflect the stream,
%   * for the MSc  degree programme "$X/Y/Z$" for a project deemed to be
%     X%, Y% and Z% of type I, II and III.
% - year              should be formatted as a 4-digit year of submission
%   (so 2014 rather than the academic year, say 2013/14 say).

\documentclass[ % the name of the author
                    author={Elizabeth Sidebottom},
                % the name of the supervisor
                supervisor={Dr. Alex Kavvos},
                % the degree programme
                    degree={MEng},
                % the dissertation    title (which cannot be blank)
                     title={Concurrency with Classical Linear Logic},
                % the dissertation subtitle (which can    be blank)
                  subtitle={},
                % the dissertation     type
                      type={programming languages},
                % the year of submission
                      year={2022}]{dissertation}

\usepackage[
  backend=biber,
  style=authoryear-icomp,
  sortlocale=de_DE,
  natbib=true,
  url=false, 
  doi=true,
  eprint=false
]{biblatex}
\addbibresource{bibo.bib}

\bibliography{dissertation}

\begin{document}

% =============================================================================

% This section simply introduces the structural guidelines.  It can clearly
% be deleted (or commented out) if you use the file as a template for your
% own dissertation: everything following it is in the correct order to use 
% as is.

\section*{Prelude}
\thispagestyle{empty}

A typical dissertation will be structured according to (somewhat) standard 
sections, described in what follows.  However, it is hard and perhaps even 
counter-productive to generalise: the goal is {\em not} to be prescriptive, 
but simply to act as a guideline.  In particular, each page count given is
important but {\em not} absolute: their aim is simply to highlight that a 
clear, concise description is better than a rambling alternative that makes
it hard to separate important content and facts from trivia.

You can use this document as a \LaTeX-based~\cite{wombat2016,lion2010} 
template for your own dissertation by simply deleting extraneous sections
and content; keep in mind that the associated {\tt Makefile} could be of
use, in particular because it automatically executes \mbox{\LaTeX} to
deal with the associated bibliography. Alternatively, upload your zip folder to Overleaf (collaborative online \LaTeX editor and compiler) 

% =============================================================================

% This macro creates the standard UoB title page by using information drawn
% from the document class (meaning it is vital you select the correct degree 
% title and so on).

\maketitle

% After the title page (which is a special case in that it is not numbered)
% comes the front matter or preliminaries; this macro signals the start of
% such content, meaning the pages are numbered with Roman numerals.

\frontmatter

% This macro creates the standard UoB declaration; on the printed hard-copy,
% this must be physically signed by the author in the space indicated.

\makedecl

% LaTeX automatically generates a table of contents, plus associated lists 
% of figures, tables and algorithms.  The former is a compulsory part of the
% dissertation, but if you do not require the latter they can be suppressed
% by simply commenting out the associated macro.

\tableofcontents
\listoffigures
\listoftables
\listofalgorithms
\lstlistoflistings

\chapter*{Abstract}

We introduce classical linear logic, a substructural logic with a classical framework.
Then we explore classical processes, a prototypical concurrent language which presents 
classical linear logic as a session typed process calculus. We then provide a translation 
from the simply typed lambda calculus into classical processes with a proof of simulation. 
This can provide the logical foundation for an enhanced message passing concurrent programming 
language which makes use of session types to provide type checking.  

\chapter*{Dedication and Acknowledgements}

I dedicate this project to Willow who saw me start this project but wasn't here to see it complete, 
and to Inca who has been with me every step of the way. 

\noindent
I would also like to thank both my supervisor, Dr. Alex Kavvos for his invaluble help with this project, 
and my boyfriend, Cam who is always supportive of everything I do.

% -----------------------------------------------------------------------------
%\chapter*{Summary of Changes}

% {\bf A conditional section, of at most $1$ page} 
% \vspace{1cm} 

% Iff. the dissertation represents a resubmission (e.g., as the result of
% a resit), this section is compulsory: the content should summarise all
% non-trivial changes made to the initial submission.  Otherwise you can
% omit it, since a summary of this type is clearly nonsensical.

% When included, the section will ideally be used to highlight additional
% work completed, and address criticism raised in any associated feedback.
% Clearly it is difficult to give generic advice about how to do so, but
% an example might be as follows:

% \begin{quote}
% \noindent
% \begin{itemize}
% \item Feedback from the initial submission criticised the design and 
%       implementation of my genetic algorithm, stating ``there seems 
%       to have been no attention to computational complexity during the
%       design, and obvious methods of optimisation are missing within
%       the resulting implementation''.  Chapter $3$ now includes a
%       comprehensive analysis of the algorithm, in terms of both time
%       and space.  While I have not altered the algorithm itself, I
%       have included a cache mechanism (also detailed in Chapter $3$)
%       that provides a significant improvement in average run-time.
% \item I added a feature in my implementation to allow automatic rather
%       than manual selection of various parameters; the experimental
%       results in Chapter $4$ have been updated to reflect this.
% \item Questions after the presentation highlighted a range of related
%       work that I had not considered: I have make a number of updates 
%       to Chapter $2$, resolving this issue.
% \end{itemize}
% \end{quote}

% -----------------------------------------------------------------------------

\chapter*{Supporting Technologies}

{\bf A compulsory section, of at most $1$ page}
\vspace{1cm} 

% \noindent
% This section should present a detailed summary, in bullet point form, 
% of any third-party resources (e.g., hardware and software components) 
% used during the project.  Use of such resources is always perfectly 
% acceptable: the goal of this section is simply to be clear about how
% and where they are used, so that a clear assessment of your work can
% result.  The content can focus on the project topic itself (rather,
% for example, than including ``I used \mbox{\LaTeX} to prepare my 
% dissertation''); an example is as follows:

% \begin{quote}
% \noindent
% \begin{itemize}
% \item I used the Java {\tt BigInteger} class to support my implementation 
%       of RSA.
% \item I used a parts of the OpenCV computer vision library to capture 
%       images from a camera, and for various standard operations (e.g., 
%       threshold, edge detection).
% \item I used an FPGA device supplied by the Department, and altered it 
%       to support an open-source UART core obtained from 
%       \url{http://opencores.org/}.
% \item The web-interface component of my system was implemented by 
%       extending the open-source WordPress software available from
%       \url{http://wordpress.org/}.
% \end{itemize}
% \end{quote}

% -----------------------------------------------------------------------------

\chapter*{Notation and Acronyms}

{\bf An optional section, of roughly $1$ or $2$ pages}
\vspace{1cm} 

% \noindent
% Any well written document will introduce notation and acronyms before
% their use, {\em even if} they are standard in some way: this ensures 
% any reader can understand the resulting self-contained content.  

% Said introduction can exist within the dissertation itself, wherever 
% that is appropriate.  For an acronym, this is typically achieved at 
% the first point of use via ``Advanced Encryption Standard (AES)'' or 
% similar, noting the capitalisation of relevant letters.  However, it 
% can be useful to include an additional, dedicated list at the start 
% of the dissertation; the advantage of doing so is that you cannot 
% mistakenly use an acronym before defining it.  A limited example is 
% as follows:

% \begin{quote}
% \noindent
% \begin{tabular}{lcl}
% AES                 &:     & Advanced Encryption Standard                                         \\
% DES                 &:     & Data Encryption Standard                                             \\
%                     &\vdots&                                                                      \\
% ${\mathcal H}( x )$ &:     & the Hamming weight of $x$                                            \\
% ${\mathbb  F}_q$    &:     & a finite field with $q$ elements                                     \\
% $x_i$               &:     & the $i$-th bit of some binary sequence $x$, st. $x_i \in \{ 0, 1 \}$ \\
% \end{tabular}
% \end{quote}

% -----------------------------------------------------------------------------

\chapter*{Acknowledgements}

{\bf An optional section, of at most $1$ page}
\vspace{1cm} 

\noindent
It is common practice (although totally optional) to acknowledge any
third-party advice, contribution or influence you have found useful
during your work.  Examples include support from friends or family, 
the input of your Supervisor and/or Advisor, external organisations 
or persons who  have supplied resources of some kind (e.g., funding, 
advice or time), and so on.

% -----------------------------------------------------------------------------
% The following sections are part of the front matter, but are not generated
% automatically by LaTeX; the use of \chapter* means they are not numbered.

% =============================================================================

% After the front matter comes a number of chapters; under each chapter,
% sections, subsections and even subsubsections are permissible.  The
% pages in this part are numbered with Arabic numerals.  Note that:
%
% - A reference point can be marked using \label{XXX}, and then later
%   referred to via \ref{XXX}; for example Chapter\ref{chap:context}.
% - The chapters are presented here in one file; this can become hard
%   to manage.  An alternative is to save the content in seprate files
%   the use \input{XXX} to import it, which acts like the #include
%   directive in C.

\mainmatter
\chapter{Introduction}
\label{chap: intro}

% The problem - no one ever wrote any basic explanation on CLL so the literature on the subject is very 
% inaccessible. The goal - to introduce CLL in an accessible format and to provide a translation from the 
% Lambda Calculus to CLL.

\noindent
There is no lack of literature on classical linear logic, however almost all of it assumes a reasonable 
level of prior knowledge on the subject. This would not be a problem if someone had written down all the 
rules and explained the purpose of CLL. Unfortunately, no one thought to do such a thing and so the rules 
of CLL are taught to students by supervisors in a one-on-one setting. The aim of this report is to present 
the rules of classical linear logic such that someone who has no prior knowledge can come away knowing 
enough to approach more literature on the topic without feeling out of depth. We will then provide a 
translation from the simply typed lambda calculus into classical linear logic which has again been 
considered by many prominent figures, but never been written out in full. 

Our main topic is classical linear logic which was first presented by Girard as an extension of classical 
logic. Classical linear logic is a classical logic without structure. It is purely mathematical and has a 
number of applications in computer science, namely in reasoning about resources and resource usage, reasoning 
about ownership, and finally in communication. The latter is seen under the Curry-Howard isomorphism, and is 
the aspect we will be focusing on.

The Curry-Howard isomorphism is a well known equivalence between mathematical proofs and computation. Here 
we use a variant of this found by Caires and Pfenning with propositions as session types, proofs as processes, 
and cut elimination as computation. This variant equates classical linear logic with a process calculus similar 
to the $\pi$ calculus and allows us to model classical linear logic as a parallel programming language. 



% {\bf A compulsory chapter, of roughly $5$ pages}

% \noindent
% This chapter should describe the project context, and motivate each of
% the proposed aims and objectives.  Ideally, it is written at a fairly 
% high-level, and easily understood by a reader who is technically 
% competent but not an expert in the topic itself.

% In short, the goal is to answer three questions for the reader.  First, 
% what is the project topic, or problem being investigated?  Second, why 
% is the topic important, or rather why should the reader care about it?  
% For example, why there is a need for this project (e.g., lack of similar 
% software or deficiency in existing software), who will benefit from the 
% project and in what way (e.g., end-users, or software developers) what 
% work does the project build on and why is the selected approach either
% important and/or interesting (e.g., fills a gap in literature, applies
% results from another field to a new problem).  Finally, what are the 
% central challenges involved and why are they significant? 
 
% The chapter should conclude with a concise bullet point list that 
% summarises the aims and objectives.  For example:

% \begin{quote}
% \noindent
% The high-level objective of this project is to reduce the performance 
% gap between hardware and software implementations of modular arithmetic.  
% More specifically, the concrete aims are:

% \begin{enumerate}
% \item Research and survey literature on public-key cryptography and
%       identify the state of the art in exponentiation algorithms.
% \item Improve the state of the art algorithm so that it can be used
%       in an effective and flexible way on constrained devices.
% \item Implement a framework for describing exponentiation algorithms
%       and populate it with suitable examples from the literature on 
%       an ARM7 platform.
% \item Use the framework to perform a study of algorithm performance
%       in terms of time and space, and show the proposed improvements
%       are worthwhile.
% \end{enumerate}
% \end{quote}

% -----------------------------------------------------------------------------
\chapter{Technical Background}
\label{chap:technical}

{\bf A compulsory chapter,     of roughly $10$ pages} 
\vspace{1cm} 

\noindent
This chapter is intended to describe the technical basis on which execution
of the project depends.  The goal is to provide a detailed explanation of
the specific problem at hand, and existing work that is relevant (e.g., an
existing algorithm that you use, alternative solutions proposed, supporting
technologies).  

Per the same advice in the handbook, note there is a subtly difference from
this and a full-blown literature review (or survey).  The latter might try
to capture and organise (e.g., categorise somehow) {\em all} related work,
potentially offering meta-analysis, whereas here the goal is simple to
ensure the dissertation is self-contained.  Put another way, after reading 
this chapter a non-expert reader should have obtained enough background to 
understand what {\em you} have done (by reading subsequent sections), then 
accurately assess your work.  You might view an additional goal as giving 
the reader confidence that you are able to absorb, understand and clearly 
communicate highly technical material.

% -----------------------------------------------------------------------------

\chapter{Classical Linear Logic}
\label{chap:execution}

Classical linear logic (CLL) is a sequent calculus which differs from classical logic in a few ways.
The first is the addition of linear negation.
In classical linear logic variables are denoted by capital letters $A, B, ...$ and the duals to these 
variables are denoted $\negg{A}, \negg{B}, ...$ where $\negg{A}$ is the negation of A such that 
$\negg{\negg{A}} = A$. This negation is not present in intuitionistic linear logic, but is a large part 
of classical linear logic and all propositions have a dual. 

The next difference is that we replace the usual two-sided sequent with a one-sided sequent, so a sequent 
in classical logic such as:
\begin{mathpar}
  A_1, A_2, ...,A_n \vdash B_1, B_2, ..., B_m
\end{mathpar}
would read as
\begin{mathpar}
  \vdash \negg{A_1}, \negg{A_2}, ..., \negg{A_n}, B_1, B_1, ..., B_m
\end{mathpar}
in classical linear logic.
This makes CLL slightly easier to read as we don't have to worry about propositions on the left-hand 
side of our sequent. We can see the use linear negation in our new one-sided sequent.

We also have that any proposition must be used exactly as many times as it appears. We cannot have some 
proposition $A$ and not make use of it, and similarly we cannot use it any more than once if it only appears 
once. 

Our propositions for CLL are as follows:

\begin{figure}[h]
  \begin{align*}
      & X &\text{variable} \\
      & \negg{X} &\text{variable dual} \\
      & A \otimes B &\text{tensor} \\
      & A \parr B &\text{par} \\
      & A \oplus B &\text{plus} \\
      & A \with B &\text{with} \\
      & \ofc A &\text{of course} \\
      & \whynot A &\text{why not} \\
      & 1 &\text{tensor unit} \\
      & \bot &\text{par unit} \\
      & 0 &\text{plus unit} \\
      & \top &\text{with unit} \\
  \end{align*}
  \caption{Propositions for Classical Linear Logic}
  \label{fig: p cll}
\end{figure}

\noindent
Generally when researching the basics of classical linear logic, one may find the vending machine example. It is 
widely used to describe the propositions of CLL in a way that is easy to understand. Say we have a vending machine 
which takes \pounds 1 coins and has the options of Tea or Coffee. If we see the tensor symbol on the machine: \emph{Tea} 
$\otimes$ \emph{Coffee}, then inserting \pounds 1 will give us both Tea and Coffee. However if we see \emph{Tea} $\with$ \emph{Coffee} then 
inserting \pounds 1 will give us a choice of either Tea or Coffee. If you can't decide what drink you want then finding a 
vending machine which says \emph{Tea} $\oplus$ \emph{Coffee} will take our \pounds 1 and give us either Tea or Coffee making the decision 
for us. Par is not quite so easy to describe, if we stay with our vending machines then we would most likely see 
$\pounds 1 \ \parr$ \emph{Tea} which would imply that inserting \pounds 1 into the vending machine will then give us Tea. 
Having a term $A \parr B$ in classical linear logic is equivalent to having a term $\negg{A} \multimap B$ in linear logic:
"not A linearly implies B".
The exponential fragment also fits in with this slightly differently. If we have a term $\ofc \pounds 1$ this would indicate 
that we have multiple \pounds 1 coins which we may use at the vending machine. We may also see a term $\whynot \pounds 1$ 
on any of these vending machines which would indicate that it is possible to use the machine more than once by inserting 
multiple \pounds 1 coins. Both of course ($\ofc$) and why not ($\whynot$) mean that whatever follows it can be used any 
number of times, or even never used at all.
\\\\
\noindent
For this report, we will modify these definitions slightly so they fit with our process calculus.

Tensor, $A \otimes B$ represents multiplicative conjunction and means output A, then continue as B. The dual to tensor 
is par which represents multiplicative disjunction. We would read $A \parr B$ as input A, then continue as B. 

We have plus for additive disjunction so $A \oplus B$ means select either A or B.  Dual to this we have with: 
$A \with B$ meaning offer a choice between A and B. 

Our exponential component consists of of course and why not, where 
$\ofc A$ means we have a server which can accept many copies of A, and $\whynot A$ means we have a client 
who may request many copies of A. 

It is important to note that every proposition has a dual, specifically:

\begin{align*}
  \negg{(A \otimes B)} &= \negg{A} \parr \negg{B} & \negg{(A \parr B)} &= \negg{A} \otimes \negg{B} \\
  \negg{(A \oplus B)} &= \negg{A} \with \negg{B} & \negg{(A \with B)} &= \negg{A} \oplus \negg{B}\\
  \negg{(\ofc A)} &= \whynot \negg{A} & \negg{(\whynot A)} &= \ofc \negg{A}
\end{align*}

\noindent
Classical linear logic naturally lends itself to parallelism, so we present the rules of CLL 
alongside a process calculus similar to $\pi$ calculus. The processes for which are as follows:

\begin{figure}[h]
  \begin{align*}
    \Proc{P} ::= & \\
    & \Proc{P \mathbin{|} Q} & \text{parallel composition} \\
    & \Proc{\Link{x}{y}} & \text{link \Name{x} with \Name{y}} \\
    & \Proc{\Out{y}{x}{P}} & \text{output \Name{x} on channel \Name{y}} \\
    & \Proc{\In{y}{x}{P}} & \text{input \Name{x} on channel \Name{y}} \\
    & \Proc{\Inl{x}{P}} & \text{left selection} \\
    & \Proc{\Inr{x}{P}} & \text{right selection} \\
    & \Proc{\Casep{x}{P}{Q}} & \text{choice} \\
    & \Proc{\New*{x}{}{P \mathbin{|} Q}} & \text{connect on channel \Name{x}} \\
    & \Proc{\whynot \Out{y}{x}{P}} & \text{client request} \\
    & \Proc{\ofc \In{y}{x}{P}} & \text{server accept} \\
    & \Proc{\Out{y}{\ }{P}} & \text{empty output} \\
    & \Proc{\In{y}{\ }{P}} & \text{empty input} \\
    & \Proc{\EmCase{x}} & \text{empty choice} \\
    & \Proc{P[\Name{x} / \Name{y}]} & \text{substitution}
  \end{align*}
  \caption{Classical Processes (CP)}
  \label{fig: p cp}
\end{figure}

\noindent
A process $\Proc{P}$ in CP can take on a few forms. Link says that we can connect two channels in 
such a way that any message delivered to $\Name{y}$ will be forwarded over $\Name{x}$ and vice verca.
For an output such as $\Proc{\Out{y}{x}{(P \mathbin{|} Q)}}$ we have that $\Name{y}$ is bound in 
$\Proc{P}$, but is not bound in $\Proc{Q}$. For the input $\Proc{\In{y}{x}{P}}$, $\Name{y}$ is bound 
in $\Proc{P}$. In $\Proc{\New*{x}{}{P \mathbin{|} Q}}$, $\Name{x}$ is bound in both $\Proc{P}$ and 
$\Proc{Q}$. Server accept is just like input so for $\Proc{\In{\ofc y}{x}{P}}$, we would have that 
$\Name{x}$ is bound in $\Proc{P}$, and similarly for $\Proc{\Out{\whynot y}{x}{P}}$. 

\section{Rules for CLL}

\noindent
We can now combine our process calculus CP with our propositions for CLL and present the rules for 
classical linear logic alongside their corresponding processes. Our typing judgements have the form 
$\IsProc{P}{\Gamma, \Name{x}: A}$ where \Proc{P} is a CP process, $\Gamma$ is a type environment, 
$\Name{x}$ is a channel, and $A$ is a CLL proposition. This typing judgement means we have some process 
\Proc{P} communicating along channel $\Name{x}$ obeying protocol $A$. Processes can communicate over 
multiple channels following different protocols, and as we saw in \ref{fig: p cp} the processes 
may also take on varying forms.

\begin{figure}[h]
  \begin{mathpar}
    \inferH{Axiom}{
    }{
      \IsProc{\Link{x}{y}}{\Name{x} : \negg{A}, \Name{y} : A}
    }\qquad
    \inferH{Cut}{
      \IsProc{P}{\Gamma, \Name{x} : \negg{A}}\\
      \IsProc{Q}{\Gamma, \Name{x} : {A}}   
    }{
      \IsProc{\New*{x}{}{P \mathbin{|} Q}}{\Gamma, \Delta}
    }\\

    \inferH{Tensor}{
        \IsProc{P}{\Gamma, \Name{x} : A} \\
        \IsProc{Q}{\Delta, \Name{y} : B}
    }{
        \IsProc{\Out{y}{x}{(P \mathbin{|} Q)}}{\Gamma, \Delta, \Name{y} : A \otimes B}
    } \qquad
    \inferH{Par}{
          \IsProc{P}{\Gamma, \Name{x} : A, \Name{y} : B} 
    }{
          \IsProc{\In{y}{x}{P}}{\Gamma, \Name{y} : A \parr B}
    }\\

    \inferH{Plus-L}{
      \IsProc{P}{\Gamma, \Name{x} : A}
    }{
      \IsProc{\Inl{x}{P}}{\Gamma, \Name{x} : A \oplus B}
    } \qquad
    \inferH{Plus-R}{
      \IsProc{P}{\Gamma, \Name{x} : B}
    }{
      \IsProc{\Inr{x}{P}}{\Gamma, \Name{x} : A \oplus B}
    }
    \qquad
    \inferH{With}{
      \IsProc{P}{\Gamma, \Name{x} : A} \\
      \IsProc{Q}{\Gamma, \Name{x} : B}
    }{
      \IsProc{\Casep{x}{P}{Q}}{\Gamma, \Name{x} : A \with B}
    }\\

    \inferH{Weakening}{
      \IsProc{P}{\Gamma}
    }{
      \IsProc{P}{\Gamma, \Name{x} : \whynot A}
    }\qquad
    \inferH{Contraction}{
      \IsProc{P}{\Gamma, \Name{x} : \whynot A, \Name{y} : \whynot A}
    }{
      \IsProc{P[\Name{x} / \Name{y}]}{\Gamma, \Name{x} : \whynot A}
    }\qquad
    \inferH{Dereliction}{
      \IsProc{P}{\Gamma, \Name{x} : A}
    }{
      \IsProc{\whynot \Out{y}{x}{P}}{\Gamma, \Name{y} : \whynot A}
    }\\

    \inferH{Promotion}{
      \IsProc{P}{\whynot \Gamma, \Name{x} : A}
    }{
      \IsProc{\ofc \In{y}{x}{P}}{\whynot \Gamma, \Name{y} : \ofc A}
    }\\

    \inferH{}{
    }{
      \IsProc{\Out{x}{\ }{P}}{\Name{x}: 1}
    }\qquad
    \inferH{}{
      \IsProc{P}{\Gamma}
    }{
      \IsProc{\In{x}{\ }{P}}{\Gamma, \Name{x}: \bot}
    }\qquad
    \inferH{}{
    }{
      \IsProc{\EmCase{x}}{\Gamma, \Name{x}: \top}
    }
  \end{mathpar}
  \caption{Rules for Classical Linear Logic with CP}
  \label{fig: r cll cp}
\end{figure}

\noindent
The \ruleref{Axiom} states that if we have some variable A, then we also have its dual $\negg{A}$. 
For the CP side of things, we see that if we have two channels following dual protocols then any 
input along \Name{x} is sent as output along \Name{y}.
The \ruleref{Cut} rule allows us to connect two processes together. As the protocols are dual, 
any transmissions and selections over one correspond with receives and choices over the other. This 
along with communication only via one channel ensures that the processes cannot get stuck.

The \ruleref{Tensor} rule outputs a fresh channel x along y, then continues as P and Q in parallel.
As P and Q communicate over different channels, we have disjoint concurrency so the processes cannot 
communicate with each other. The new process, $\Proc{P \mathbin{|} Q}$ communicates over channel \Name{y}.
The rule \ruleref{Par} inputs A, then continues as B. This is known as connected concurrency as P can 
communicate along both \Name{x} and \Name{y}.

The rule \ruleref{Plus-L} indicates left selection, and similarly \ruleref{Plus-R} indicates right 
selection. The process $\Proc{\Inl{x}{P}}$ obeys protocol $A \oplus B$ by requesting the left option 
from a choice sent along $x$. The process for $\Proc{\Inr{x}{P}}$ is symmetric. 
The \ruleref{With} rule offers a choice between processes \Proc{P} and \Proc{Q}. The new process 
$\Proc{\Casep{x}{P}{Q}}$ will receive a selection over channel $x$ and execute either \Proc{P} or 
\Proc{Q} accordingly.

\ruleref{Weakening} lets us consider a process which doesn't communicate or follow a protocol to be 
a process which communicates along a channel $\Name{x}$ with protocol \whynot A. 
If a process \Proc{P} communicates along two channels, both following the same protocol \whynot A, 
then we can use \ruleref{Contraction} to substitute one channel for another so \Proc{P} communicates 
along only one channel following protocol \whynot A.
\ruleref{Dereliction} allows us to.

\section{Cut Reduction}

\noindent 
Just like in the simply typed lambda calculus, there are dynamic rules for classical linear logic. 
We have already seen the \ruleref{Cut} rule and the dynamics for CLL make use of this rule to simplify 
terms via cut reduction. 

\begin{mathpar}

  \inferH{}{
    \inferH{}{
    }{
      \IsProc{\Link{x}{y}}{\Name{x}: \negg{A}, \Name{y}: A}
    }\\
    \inferH{}{
    }{
      \IsProc{P}{\Gamma, \Name{x}: A}
    }
  }{
    \IsProc{\New*{x}{}{\Link{x}{y} \mathbin{|} P}}{\Gamma, \Name{y}: A}
  } \quad \Longrightarrow \quad
  \IsProc{P[\Name{y} / \Name{x}]}{\Gamma, \Name{y}: A} \\\\

  \inferH{}{
    \inferH{}{
      \IsProc{P}{\Gamma, \Name{x} : \negg{A}} \\
      \IsProc{Q}{\Delta, \Name{y} : \negg{B}}
    }{
      \IsProc{\Out{y}{x}{(P \mathbin{|} Q)}}{\Gamma, \Delta, \Name{y} : \negg{A} \otimes \negg{B}} \\
    }\\
    \inferH{}{
      \IsProc{R}{\Theta, \Name{x} : A, \Name{y} : B}
    }{
      \IsProc{\In{y}{x}{R}}{\Theta, \Name{y} : A \parr B}
    }
  }{
    \IsProc{\New*{y}{}{\Out{y}{x}{(P \mathbin{|} Q)} \mathbin{|} \In{y}{x}{R}}}{\Gamma, \Delta, \Theta}
  } \quad \Longrightarrow \\

  \inferH{}{
    \inferH{}{
      \IsProc{P}{\Gamma, \Name{x} : \negg{A}} \\
      \IsProc{R}{\Theta, \Name{x} : A, \Name{y}: B}
    }{
      \IsProc{\New*{x}{}{P \mathbin{|} Q}}{\Gamma, \Theta, \Name{y} : B} \\
    }\\
    \inferH{}{
    }{
      \IsProc{Q}{\Delta, \Name{y} : \negg{B}}
    }
  }{
    \IsProc{\New*{y}{}{\New*{x}{}{P \mathbin{|} R} \mathbin{|} Q}}{\Gamma, \Delta, \Theta}
  }\\\\

  \inferH{}{
    \inferH{}{
      \IsProc{P}{\Gamma, \Name{x}: A}
    }{
      \IsProc{\Inl{x}{P}}{\Gamma, \Name{x}: A \oplus B}
    }\\
    \inferH{}{
      \IsProc{Q}{\Delta, \Name{x}: \negg{A}}\\
      \IsProc{R}{\Theta, \Name{x}: \negg{B}}
    }{
      \IsProc{\Casep{x}{Q}{R}}{\Delta, \Name{x}: A \with B}
    }
  }{
    \IsProc{\New*{x}{}{\Inl{x}{P} \mathbin{|} \Casep{x}{Q}{R}}}{\Gamma, \Delta}
  } \quad \Longrightarrow \quad
  \inferH{}{
    \IsProc{P}{\Gamma, \Name{x}: A}\\
    \IsProc{Q}{\Delta, \Name{x}: \negg{A}}
  }{
    \IsProc{\New*{x}{}{P \mathbin{|} Q}}{\Gamma, \Delta}
  }\\\\

  \inferH{}{
    \inferH{}{
      \IsProc{P}{\whynot \Gamma, \Name{x}: A}
    }{
      \IsProc{\ofc \In{y}{x}{P}}{\whynot \Gamma, \Name{y}: \ofc A}
    }\\
    \inferH{}{
      \IsProc{Q}{\Delta, \Name{x}: \negg{A}}
    }{
      \IsProc{\whynot \Out{y}{x}{Q}}{\Delta, \Name{y}: \negg{\whynot A}}
    }
  }{
    \IsProc{\New*{y}{}{\ofc \In{y}{x}{P} \mathbin{|} \whynot \Out{y}{x}{Q}}}{\whynot \Gamma, \Delta}
  } \quad \Longrightarrow \quad
  \inferH{}{
    \IsProc{P}{\whynot \Gamma, \Name{x}: A}\\
    \IsProc{Q}{\Delta, \Name{x}: \negg{A}}
  }{
    \IsProc{\New*{x}{}{P \mathbin{|} Q}}{\whynot \Gamma, \Delta}
  }\\\\

  \inferH{}{
    \inferH{}{
      \IsProc{P}{\whynot \Gamma, \Name{x}: A}
    }{
      \IsProc{\ofc \In{y}{x}{P}}{\whynot \Gamma, \Name{y}: \ofc A}
    }\\
    \inferH{}{
      \IsProc{Q}{\Delta}
    }{
      \IsProc{Q}{\Delta, \Name{x}: \negg{\whynot A}}
    }
  }{
    \IsProc{\New*{y}{}{\ofc \In{y}{x}{P} \mathbin{|} Q}}{\whynot \Gamma, \Delta}
  } \quad \Longrightarrow \quad
  \inferH{}{
    \IsProc{Q}{\Delta}
  }{
    \IsProc{Q}{\whynot \Gamma, \Delta}
  }\\\\

  \inferH{}{
    \inferH{}{
      \IsProc{P}{\whynot \Gamma, \Name{x}: A}
    }{
      \IsProc{\ofc \In{y}{x}{P}}{\whynot \Gamma, \Name{y}: \ofc A}
    }\\
    \inferH{}{
      \IsProc{Q}{\Delta, \Name{y}: \whynot A, \Name{z}: \whynot A}
    }{
      \IsProc{Q[\Name{y} / \Name{z}]}{\Delta, \Name{y}: \whynot A}
    }
  }{
    \IsProc{\New*{y}{}{\ofc \In{y}{x}{P} \mathbin{|} Q[\Name{y} / \Name{z}]}}{\whynot \Gamma, \Delta}
  }\quad \Longrightarrow \\

  \inferH{}{
    \inferH{}{
      \IsProc{P}{\whynot \Gamma, \Name{x}: A}
    }{
      \IsProc{\ofc \In{y}{x}{P}}{\whynot \Gamma, \Name{y}: \ofc A}
    }\\
    \inferH{}{
      \inferH{}{
        \IsProc{P'}{\whynot \Gamma', \Name{w}: A}
      }{
        \IsProc{\ofc \In{z}{w}{P}}{\whynot \Gamma', \Name{z}: \ofc A}
      }\\
      \IsProc{Q}{\Delta, \Name{y}: \negg{\whynot A}, \Name{z}: \negg{\whynot A}}
    }{
      \IsProc{\New*{z}{}{\ofc \In{z}{w}{P} \mathbin{|} Q}}{\whynot \Gamma', \Delta, \Name{x}: \negg{\whynot A}}
    }
  }{
    \inferH{}{
      \IsProc{\New*{y}{}{\ofc \In{y}{x}{P} \mathbin{|} \New*{z}{}{\ofc \In{z}{w}{P'} \mathbin{|} Q}}}{\whynot \Gamma, \whynot \Gamma', \Delta}
    }{
      \IsProc{\New*{y}{}{\ofc \In{y}{x}{P} \mathbin{|} \New*{z}{}{\ofc \In{z}{y}{P} \mathbin{|} Q}}}{\whynot \Gamma, \Delta}
    }
  }\\\\

  \inferH{}{
    \inferH{}{
      \IsProc{P}{\Gamma}
    }{
      \IsProc{\In{x}{\ }{P}}{\Gamma, \Name{x}: \bot}
    }\\
    \inferH{}{
    }{
      \IsProc{\Out{x}{\ }{Q}}{\Name{x}: 1}
    }
  }{
    \IsProc{\New*{x}{}{\ }\In{x}{\ }{P} \mathbin{|} \Out{x}{\ }{Q}}{\Gamma}
  }\quad \Longrightarrow \quad
  \IsProc{P}{\Gamma}
\end{mathpar}





% {\bf A topic-specific chapter, of roughly $15$ pages} 
% \vspace{1cm} 

% \noindent
% This chapter is intended to describe what you did: the goal is to explain
% the main activity or activities, of any type, which constituted your work 
% during the project.  The content is highly topic-specific, but for many 
% projects it will make sense to split the chapter into two sections: one 
% will discuss the design of something (e.g., some hardware or software, or 
% an algorithm, or experiment), including any rationale or decisions made, 
% and the other will discuss how this design was realised via some form of 
% implementation.  

% This is, of course, far from ideal for {\em many} project topics.  Some
% situations which clearly require a different approach include:

% \begin{itemize}
% \item In a project where asymptotic analysis of some algorithm is the goal,
%       there is no real ``design and implementation'' in a traditional sense
%       even though the activity of analysis is clearly within the remit of
%       this chapter.
% \item In a project where analysis of some results is as major, or a more
%       major goal than the implementation that produced them, it might be
%       sensible to merge this chapter with the next one: the main activity 
%       is such that discussion of the results cannot be viewed separately.
% \end{itemize}

% \noindent
% Note that it is common to include evidence of ``best practice'' project 
% management (e.g., use of version control, choice of programming language 
% and so on).  Rather than simply a rote list, make sure any such content 
% is useful and/or informative in some way: for example, if there was a 
% decision to be made then explain the trade-offs and implications 
% involved.

% -----------------------------------------------------------------------------

\chapter{Translation}

Here we present a translation from the simply typed lambda calculus into classical linear logic. First 
we will give the direct translation of each type, then a detailed derivation of how we got there. We 
make use of the $^{\circ}$ symbol to denote a translation. On the left hand side we have the lambda 
calculus type, and on the right we have its classical linear logic translation.

\begin{align*}
    (1)^\circ &= \top^\circ \\
    (A + B)^\circ &= A^\circ \oplus B^\circ \\
    (A \times B)^\circ &= A^\circ \otimes B^\circ \\
    (A \rightarrow B)^\circ &= \negg{(\ofc A^\circ)} \parr B^\circ
\end{align*}

The translation of some environment $\Gamma$ are as follows:

\begin{align*}
  \Gamma^\circ & \\
  (\cdot)^\circ &= \cdot \\
  (\Gamma, x:A)^\circ &= \Gamma^\circ, x: \negg{(\ofc A^\circ)} \\
   &= \Gamma^\circ, x: \whynot \negg{(A^\circ)}
\end{align*}

We will also need a derivable rule, \ruleref{Depar} which is as follows:

\begin{mathpar}
  \inferH{}{
    \inferH{}{
      \IsProc{\Link{w}{x}}{\Name{x}: A, \Name{w}: \negg{A}}\\
      \IsProc{\Link{y}{z}}{\Name{y}: B, \Name{z}: \negg{B}}
    }{
      \IsProc{\Out{z}{w}{(\Link{w}{x} \mathbin{|} \Link{y}{z})}}{\Name{x}: A, \Name{y}: B, \Name{z}: \negg{A} \otimes \negg{B}}
    }\\
    \inferH{}{
    }{
      \IsProc{P}{\Gamma, \Name{z}: A \parr B}
    }
  }{
    \IsProc{\New*{z}{}{\Out{z}{w}{(\Link{w}{x} \mathbin{|} \Link{y}{z})} \mathbin{|} P}}{\Gamma, \Name{x}: A, \Name{y}: B}
  }\quad \leadsto \quad
  \inferH{Depar}{
    \IsProc{P}{\Gamma, \Name{z}: A \parr B}
  }{
    \IsProc{\Depar{x}{y}{P}}{\Gamma, \Name{x}: A, \Name{y}: B}
  }
\end{mathpar}

\noindent
\ruleref{Depar} takes a concurrent process $A \parr B$ and disconnects them to give us 
both $A$ and $B$ individually.

\begin{mathpar}
    %var
    {\Biggl\llbracket
    \inferH{Var}{
    }{
        \Gamma, x: A \vdash x: A
    }\Biggr\rrbracket}_\Name{x}
    \EqDef
    \inferH{}{
        \inferH{}{
        \IsProc{\Link{x}{y}}{\Name{x}: \negg{(A^\circ)}, \Name{y}: A^\circ}
        }{
            \IsProc{\whynot \Out{z}{x}{\Link{x}{y}}}{\Name{z}: \whynot \negg{(A^\circ)}, \Name{y}: A^\circ}
        }
    }{
        \IsProc{\whynot \Out{z}{x}{\Link{x}{y}}}{\Gamma^\circ \Name{z}: \whynot \negg{(A^\circ)}, \Name{y}: A^\circ}
    }\\

    %unit
    {\Biggl\llbracket
    \inferH{Unit}{
    }{
    \Gamma \vdash \langle \rangle : 1
    }\Biggr\rrbracket}_\Name{x}
    \EqDef
    \inferH{}{
    }{
    \IsProc{\EmCase{x}}{\Gamma^{\circ}, \Name{x} : \top^{\circ}}
    }\\

    %left injection
    {\Biggl\llbracket
    \inferH{In-l}{
      \Gamma \vdash M:A
    }{
      \Gamma \vdash \LamInl{M}: A+B
    }\Biggr\rrbracket}_\Name{x}
    \EqDef
    \inferH{}{
      \IsProc{\llbracket M \rrbracket_\Name{x}}{\Gamma^\circ, \Name{x}: A^\circ}
    }{
      \IsProc{\Inl{x}{\llbracket M \rrbracket_\Name{x}}}{\Gamma^\circ, \Name{x}: A^\circ \oplus B^\circ}
    }\\

    %right injection
    {\Biggl\llbracket
    \inferH{In-r}{
      \Gamma \vdash M:B
    }{
      \Gamma \vdash \LamInr{M}: A+B
    }\Biggr\rrbracket}_\Name{x}
    \EqDef
    \inferH{}{
      \IsProc{\llbracket M \rrbracket_\Name{x}}{\Gamma^\circ, \Name{x}: B^\circ}
    }{
      \IsProc{\Inr{x}{\llbracket M \rrbracket_\Name{x}}}{\Gamma^\circ, \Name{x}: A^\circ \oplus B^\circ}
    }\\

    %here we use the anti barendregt convention for convenience
    %case
    \Biggl\llbracket
    \inferH{Case}{
      \Gamma \vdash M: A+B \\
      \Gamma, x:A \vdash P:C \\
      \Gamma, x:B \vdash Q:C
    }{
      \Gamma \vdash \LamCase{M}{x}{P}{x}{Q}:C
    }\Biggr\rrbracket_\Name{x}
    \EqDef
    \inferH{}{
      \IsProc{\llbracket M \rrbracket_\Name{x}}{\Gamma^\circ, \Name{x}: A^\circ \oplus B^\circ} \\
      \IsProc{\llbracket P \rrbracket_\Name{y}}{\Gamma^\circ, \Name{y}: C^\circ, \Name{x}: \negg{(A^\circ)}} \\
      \IsProc{\llbracket Q \rrbracket_\Name{y}}{\Gamma^\circ, \Name{y}: C^\circ, \Name{x}: \negg{(B^\circ)}}
    }{
        \inferH{}{
        \IsProc{\llbracket M \rrbracket_\Name{x}}{\Gamma^\circ, \Name{x}: A^\circ \oplus B^\circ} \\
        \IsProc{\Casep{x}{\llbracket P \rrbracket_\Name{y}}{\llbracket Q \rrbracket_\Name{y}}}{\Gamma^\circ, \Name{y}: C^\circ, \Name{x}: \negg{(A^\circ)} \with \negg{(B^\circ)}}
      }{
          \inferH{}{
          \IsProc{\New*{x}{}{\llbracket M \rrbracket_\Name{x} \mathbin{|} \Casep{x}{\llbracket P \rrbracket_\Name{y}}{\llbracket Q \rrbracket_\Name{y}}}}{\Gamma^\circ, \Name{y}: C^\circ}
        }{
          \IsProc{\New*{x}{}{\llbracket M \rrbracket_\Name{x} \mathbin{|} \Casep{x}{\llbracket P \rrbracket_\Name{y}}{\llbracket Q \rrbracket_\Name{y}}}}{\Gamma^\circ, \Name{y}: C^\circ}
    }}}

    %product
    {\Biggl\llbracket
    \inferH{Prod}{
    \Gamma \vdash M : A \\
    \Gamma \vdash N : B
    }{
    \Gamma \vdash \Tuple{M}{N} : A \times B
    }\Biggr\rrbracket}_{\Name{x}}
    \EqDef
    \inferH{}{
    \IsProc{\llbracket M \rrbracket_\Name{x}}{\Gamma^{\circ}, \Name{x} : A^{\circ}} \\
    \IsProc{\llbracket N \rrbracket_\Name{x}}{\Gamma^{\circ}, \Name{x} : B^{\circ}}
    }{
    \IsProc{\Casep{x}{{\llbracket M \rrbracket}_\Name{x}}{{\llbracket N \rrbracket}_\Name{x}}}{\Gamma^{\circ}, \Name{x} : A^{\circ} \with B^{\circ}}
    }\\

    %left projection
    {\Biggl\llbracket
    \inferH{Prj-l}{
    \Gamma \vdash M : A \times B
    }{
    \Gamma \vdash \Proj{M} : A
    }\Biggr\rrbracket}_{\Name{x}}
    \EqDef
    \inferH{}{
    \inferH{}{
        \IsProc{\Link{x}{y}}{\Name{x} : A, \Name{y} : \negg{(A^{\circ})}}
    }{
    \IsProc{\Inl{y}{\Link{x}{y}}}{\Name{x} : A^{\circ}, \Name{y} : \negg{(A^{\circ})} \oplus \negg{(B^{\circ})}} \\
    }\\
    \inferH{}{
    }{
        \IsProc{{\llbracket M \rrbracket}_\Name{y}}{\Gamma^{\circ}, \Name{y} : A^{\circ} \with B^{\circ}}
    }}{
    \IsProc{\New*{y}{}{\Inl{y}{\Link{y}{x}} \mathbin{|} {\llbracket M \rrbracket}_\Name{y}}}{\Gamma^{\circ}, \Name{x} : A^{\circ}}
    }\\

    %right projection
    {\Biggl\llbracket
    \inferH{Prj-r}{
    \Gamma \vdash M : A \times B
    }{
    \Gamma \vdash \Proj[2]{M} : B
    }\Biggr\rrbracket}_{\Name{x}}
    \EqDef
    \inferH{}{
    \inferH{}{
        \IsProc{\Link{x}{y}}{\Name{x} : B, \Name{y} : \negg{(B^{\circ})}}
    }{
    \IsProc{\Inl{y}{\Link{x}{y}}}{\Name{x} : B^{\circ}, \Name{y} : \negg{(A^{\circ})} \oplus \negg{(B^{\circ})}} \\
    }\\
    \inferH{}{
    }{
        \IsProc{{\llbracket M \rrbracket}_\Name{y}}{\Gamma^{\circ}, \Name{y} : A^{\circ} \with B^{\circ}}
    }}{
    \IsProc{\New*{y}{}{\Inl{y}{\Link{y}{x}} \mathbin{|} {\llbracket M \rrbracket}_\Name{y}}}{\Gamma^{\circ}, \Name{x} : B^{\circ}}
    }\\

    %lambda abstraction
    {\Biggl\llbracket 
    \inferH{Lam}{
    \Gamma, x :A \vdash M :B
    }{
    \Gamma \vdash \Lam{x}[A]{M} : A \rightarrow B
    } \Biggr\rrbracket}_{\Name{y}}
    \EqDef
    \inferH{}{
    \IsProc{{\llbracket M \rrbracket}_\Name{y}}{\Gamma^{\circ}, \Name{x} : \negg{(\ofc A^{\circ})}, \Name{y} : B^{\circ}}
    }{
    \IsProc{\In{y}{x}{{\llbracket M \rrbracket}_\Name{y}}}{\Gamma^{\circ}, \Name{y} : \negg{(\ofc A^{\circ})} \parr B^{\circ}}
    }\\

    %application
    \Biggl\llbracket 
    \inferH{App}{
      \Gamma \vdash M:A \rightarrow B \\
      \Gamma \vdash N:A
    }{
      \Gamma \vdash MN:B
    }\Biggr\rrbracket_\Name{y}
    \EqDef
    \inferH{}{
      \inferH{}{
        \IsProc{\llbracket M \rrbracket_\Name{x}}{\Gamma^\circ, \Name{x}: \negg{(\ofc A^\circ)} \parr B^\circ}
      }{
        \IsProc{\Depar{x}{y}{\llbracket M \rrbracket_\Name{x}}}{\Gamma^\circ, \Name{x}: \negg{(\ofc A^\circ)}, \Name{y}: B^\circ}
      }\\
      \inferH{}{
        \IsProc{\llbracket N \rrbracket_\Name{y}}{\Gamma^\dagger, \Name{y^\dagger}: A^\circ}
      }{
        \IsProc{\ofc \In{x}{y}{\llbracket N \rrbracket_\Name{y}}}{\Gamma^\dagger, \Name{x^\dagger}: \ofc A^\circ}
      }
    }{
      \IsProc{\New*{x}{}{\Depar{x}{y}{\llbracket M \rrbracket_\Name{x}} \mathbin{|} \ofc \In{x}{y}{\llbracket N \rrbracket_\Name{y}}}}{\Gamma^\circ, \Gamma^\dagger, \Name{y}: B^\circ}
    }
\end{mathpar}

It is interesting to note that we have an inversion of polarity from the lambda calculus 
to classical linear logic. All type constructors in the lambda calculus become destructors 
in CLL and vice verca. 

For the translation of the case rule, we use the anti-Barendregt convention and conveniently 
name variables in such a way that we do not have to rename them in the future. 

% -----------------------------------------------------------------------------
\chapter{Critical Evaluation}
\label{chap:evaluation}

{\bf A topic-specific chapter, of roughly $15$ pages} 
\vspace{1cm} 

\noindent
This chapter is intended to evaluate what you did.  The content is highly 
topic-specific, but for many projects will have flavours of the following:

\begin{enumerate}
\item functional  testing, including analysis and explanation of failure 
      cases,
\item behavioural testing, often including analysis of any results that 
      draw some form of conclusion wrt. the aims and objectives,
      and
\item evaluation of options and decisions within the project, and/or a
      comparison with alternatives.
\end{enumerate}

\noindent
This chapter often acts to differentiate project quality: even if the work
completed is of a high technical quality, critical yet objective evaluation 
and comparison of the outcomes is crucial.  In essence, the reader wants to
learn something, so the worst examples amount to simple statements of fact 
(e.g., ``graph X shows the result is Y''); the best examples are analytical 
and exploratory (e.g., ``graph X shows the result is Y, which means Z; this 
contradicts [1], which may be because I use a different assumption'').  As 
such, both positive {\em and} negative outcomes are valid {\em if} presented 
in a suitable manner.

% -----------------------------------------------------------------------------
\chapter{Conclusion}
\label{chap:conclusion}

{\bf A compulsory chapter,     of roughly $5$ pages} 
\vspace{1cm} 

% \noindent
% The concluding chapter of a dissertation is often underutilised because it 
% is too often left too close to the deadline: it is important to allocation
% enough attention.  Ideally, the chapter will consist of three parts:

% \begin{enumerate}
% \item (Re)summarise the main contributions and achievements, in essence
%       summing up the content.
% \item Clearly state the current project status (e.g., ``X is working, Y 
%       is not'') and evaluate what has been achieved with respect to the 
%       initial aims and objectives (e.g., ``I completed aim X outlined 
%       previously, the evidence for this is within Chapter Y'').  There 
%       is no problem including aims which were not completed, but it is 
%       important to evaluate and/or justify why this is the case.
% \item Outline any open problems or future plans.  Rather than treat this
%       only as an exercise in what you {\em could} have done given more 
%       time, try to focus on any unexplored options or interesting outcomes
%       (e.g., ``my experiment for X gave counter-intuitive results, this 
%       could be because Y and would form an interesting area for further 
%       study'' or ``users found feature Z of my software difficult to use,
%       which is obvious in hindsight but not during at design stage; to 
%       resolve this, I could clearly apply the technique of Smith [7]'').
% \end{enumerate}

% -----------------------------------------------------------------------------
% =============================================================================

% Finally, after the main matter, the back matter is specified.  This is
% typically populated with just the bibliography.  LaTeX deals with these
% in one of two ways, namely
%
% - inline, which roughly means the author specifies entries using the 
%   \bibitem macro and typesets them manually, or
% - using BiBTeX, which means entries are contained in a separate file
%   (which is essentially a databased) then inported; this is the 
%   approach used below, with the databased being dissertation.bib.
%
% Either way, the each entry has a key (or identifier) which can be used
% in the main matter to cite it, e.g., \cite{X}, \cite[Chapter 2}{Y}.

\backmatter

\printbibliography

% The dissertation concludes with a set of (optional) appendicies; these are 
% the same as chapters in a sense, but once signaled as being appendicies via
% the associated macro, LaTeX manages them appropriatly.

\appendix

\chapter{An Example Appendix}
\label{appx:example}

Content which is not central to, but may enhance the dissertation can be 
included in one or more appendices; examples include, but are not limited
to

\begin{itemize}
\item lengthy mathematical proofs, numerical or graphical results which 
      are summarised in the main body,
\item sample or example calculations, 
      and
\item results of user studies or questionnaires.
\end{itemize}

\noindent
Note that in line with most research conferences, the marking panel is not
obliged to read such appendices.

% -----------------------------------------------------------------------------
% =============================================================================

\end{document}

% The document class supplies options to control rendering of some standard
% features in the result.  The goal is for uniform style, so some attention 
% to detail is *vital* with all fields.  Each field (i.e., text inside the
% curly braces below, so the MEng text inside {MEng} for instance) should 
% take into account the following:
%
% - author name       should be formatted as "FirstName LastName"
%   (not "Initial LastName" for example),
% - supervisor name   should be formatted as "Title FirstName LastName"
%   (where Title is "Dr." or "Prof." for example),
% - degree programme  should be "BSc", "MEng", "MSci", "MSc" or "PhD",
% - dissertation title should be correctly capitalised (plus you can have
%   an optional sub-title if appropriate, or leave this field blank),
% - dissertation type should be formatted as one of the following:
%   * for the MEng degree programme either "enterprise" or "research" to
%     reflect the stream,
%   * for the MSc  degree programme "$X/Y/Z$" for a project deemed to be
%     X%, Y% and Z% of type I, II and III.
% - year              should be formatted as a 4-digit year of submission
%   (so 2014 rather than the academic year, say 2013/14 say).

\documentclass[ % the name of the author
                    author={Elizabeth Sidebottom},
                % the name of the supervisor
                supervisor={Dr. Alex Kavvos},
                % the degree programme
                    degree={MEng},
                % the dissertation title (which cannot be blank)
                     title={Concurrency with Classical Linear Logic},
                % the dissertation subtitle (which can    be blank)
                  subtitle={},
                % the dissertation type
                      type={Programming Languages},
                % the year of submission
                      year={2022}]{dissertation}

\usepackage[
  backend=biber,
  % style=authoryear-icomp,
  % sortlocale=de_DE,
  natbib=true,
  url=false, 
  doi=true,
  eprint=false
]{biblatex}
\addbibresource{bibo.bib}

\usepackage{mathpartir}
\usepackage{stmaryrd}
\usepackage{locallabel}
\usepackage{pftools}
\usepackage{pi-macros}
\usepackage{lambda-macros}
% \usepackage{ebproof}

\newcommand\EqDef{\stackrel{\text{def}}{=}}

\bibliography{dissertation}

\begin{document}

% =============================================================================

% This section simply introduces the structural guidelines.  It can clearly
% be deleted (or commented out) if you use the file as a template for your
% own dissertation: everything following it is in the correct order to use 
% as is.

% \section*{Prelude}
% \thispagestyle{empty}

% A typical dissertation will be structured according to (somewhat) standard 
% sections, described in what follows.  However, it is hard and perhaps even 
% counter-productive to generalise: the goal is {\em not} to be prescriptive, 
% but simply to act as a guideline.  In particular, each page count given is
% important but {\em not} absolute: their aim is simply to highlight that a 
% clear, concise description is better than a rambling alternative that makes
% it hard to separate important content and facts from trivia.

% You can use this document as a \LaTeX-based~\cite{wombat2016,lion2010} 
% template for your own dissertation by simply deleting extraneous sections
% and content; keep in mind that the associated {\tt Makefile} could be of
% use, in particular because it automatically executes \mbox{\LaTeX} to
% deal with the associated bibliography. Alternatively, upload your zip folder 
% to Overleaf (collaborative online \LaTeX editor and compiler) 

% =============================================================================

% This macro creates the standard UoB title page by using information drawn
% from the document class (meaning it is vital you select the correct degree 
% title and so on).

\maketitle

% After the title page (which is a special case in that it is not numbered)
% comes the front matter or preliminaries; this macro signals the start of
% such content, meaning the pages are numbered with Roman numerals.

\frontmatter

% This macro creates the standard UoB declaration; on the printed hard-copy,
% this must be physically signed by the author in the space indicated.

\makedecl

% LaTeX automatically generates a table of contents, plus associated lists 
% of figures, tables and algorithms.  The former is a compulsory part of the
% dissertation, but if you do not require the latter they can be suppressed
% by simply commenting out the associated macro.

\tableofcontents
% \listoffigures
% \listoftables
% \listofalgorithms
% \lstlistoflistings

\chapter*{Abstract}

We introduce classical linear logic, a substructural logic with a classical framework.
Then we explore classical processes, a prototypical concurrent language which presents 
classical linear logic as a session typed process calculus. We then provide a translation 
from the simply typed lambda calculus into classical processes with a proof of simulation. 
This can provide the logical foundation for an enhanced message passing concurrent programming 
language which makes use of session types to provide type checking.  

\chapter*{Dedication and Acknowledgements}

I dedicate this project to Willow who saw me start this project but wasn't here to see it complete, 
and to Inca who has been with me every step of the way. 

\noindent
I would also like to thank both my supervisor, Dr. Alex Kavvos for his invaluble help with this project, 
and my boyfriend, Cam who is always supportive of everything I do.

% -----------------------------------------------------------------------------
% %\chapter*{Summary of Changes}

% {\bf A conditional section, of at most $1$ page} 
% \vspace{1cm} 

% Iff. the dissertation represents a resubmission (e.g., as the result of
% a resit), this section is compulsory: the content should summarise all
% non-trivial changes made to the initial submission.  Otherwise you can
% omit it, since a summary of this type is clearly nonsensical.

% When included, the section will ideally be used to highlight additional
% work completed, and address criticism raised in any associated feedback.
% Clearly it is difficult to give generic advice about how to do so, but
% an example might be as follows:

% \begin{quote}
% \noindent
% \begin{itemize}
% \item Feedback from the initial submission criticised the design and 
%       implementation of my genetic algorithm, stating ``there seems 
%       to have been no attention to computational complexity during the
%       design, and obvious methods of optimisation are missing within
%       the resulting implementation''.  Chapter $3$ now includes a
%       comprehensive analysis of the algorithm, in terms of both time
%       and space.  While I have not altered the algorithm itself, I
%       have included a cache mechanism (also detailed in Chapter $3$)
%       that provides a significant improvement in average run-time.
% \item I added a feature in my implementation to allow automatic rather
%       than manual selection of various parameters; the experimental
%       results in Chapter $4$ have been updated to reflect this.
% \item Questions after the presentation highlighted a range of related
%       work that I had not considered: I have make a number of updates 
%       to Chapter $2$, resolving this issue.
% \end{itemize}
% \end{quote}

% -----------------------------------------------------------------------------

% \chapter*{Supporting Technologies}

{\bf A compulsory section, of at most $1$ page}
\vspace{1cm} 

% \noindent
% This section should present a detailed summary, in bullet point form, 
% of any third-party resources (e.g., hardware and software components) 
% used during the project.  Use of such resources is always perfectly 
% acceptable: the goal of this section is simply to be clear about how
% and where they are used, so that a clear assessment of your work can
% result.  The content can focus on the project topic itself (rather,
% for example, than including ``I used \mbox{\LaTeX} to prepare my 
% dissertation''); an example is as follows:

% \begin{quote}
% \noindent
% \begin{itemize}
% \item I used the Java {\tt BigInteger} class to support my implementation 
%       of RSA.
% \item I used a parts of the OpenCV computer vision library to capture 
%       images from a camera, and for various standard operations (e.g., 
%       threshold, edge detection).
% \item I used an FPGA device supplied by the Department, and altered it 
%       to support an open-source UART core obtained from 
%       \url{http://opencores.org/}.
% \item The web-interface component of my system was implemented by 
%       extending the open-source WordPress software available from
%       \url{http://wordpress.org/}.
% \end{itemize}
% \end{quote}

% -----------------------------------------------------------------------------

% \chapter*{Notation and Acronyms}

{\bf An optional section, of roughly $1$ or $2$ pages}
\vspace{1cm} 

% \noindent
% Any well written document will introduce notation and acronyms before
% their use, {\em even if} they are standard in some way: this ensures 
% any reader can understand the resulting self-contained content.  

% Said introduction can exist within the dissertation itself, wherever 
% that is appropriate.  For an acronym, this is typically achieved at 
% the first point of use via ``Advanced Encryption Standard (AES)'' or 
% similar, noting the capitalisation of relevant letters.  However, it 
% can be useful to include an additional, dedicated list at the start 
% of the dissertation; the advantage of doing so is that you cannot 
% mistakenly use an acronym before defining it.  A limited example is 
% as follows:

% \begin{quote}
% \noindent
% \begin{tabular}{lcl}
% AES                 &:     & Advanced Encryption Standard                                         \\
% DES                 &:     & Data Encryption Standard                                             \\
%                     &\vdots&                                                                      \\
% ${\mathcal H}( x )$ &:     & the Hamming weight of $x$                                            \\
% ${\mathbb  F}_q$    &:     & a finite field with $q$ elements                                     \\
% $x_i$               &:     & the $i$-th bit of some binary sequence $x$, st. $x_i \in \{ 0, 1 \}$ \\
% \end{tabular}
% \end{quote}

% -----------------------------------------------------------------------------

% \chapter*{Acknowledgements}

{\bf An optional section, of at most $1$ page}
\vspace{1cm} 

\noindent
It is common practice (although totally optional) to acknowledge any
third-party advice, contribution or influence you have found useful
during your work.  Examples include support from friends or family, 
the input of your Supervisor and/or Advisor, external organisations 
or persons who  have supplied resources of some kind (e.g., funding, 
advice or time), and so on.

% -----------------------------------------------------------------------------
% The following sections are part of the front matter, but are not generated
% automatically by LaTeX; the use of \chapter* means they are not numbered.

% =============================================================================

% After the front matter comes a number of chapters; under each chapter,
% sections, subsections and even subsubsections are permissible.  The
% pages in this part are numbered with Arabic numerals.  Note that:
%
% - A reference point can be marked using \label{XXX}, and then later
%   referred to via \ref{XXX}; for example Chapter\ref{chap:context}.
% - The chapters are presented here in one file; this can become hard
%   to manage.  An alternative is to save the content in seprate files
%   the use \input{XXX} to import it, which acts like the #include
%   directive in C.

\mainmatter
\chapter{Introduction}
\label{chap: intro}

% The problem - no one ever wrote any basic explanation on CLL so the literature on the subject is very 
% inaccessible. The goal - to introduce CLL in an accessible format and to provide a translation from the 
% Lambda Calculus to CLL.

\noindent
There is no lack of literature on classical linear logic (CLL), however almost all of it assumes a reasonable 
level of prior knowledge on the subject. This would not be a problem if someone had written down all the 
rules and explained the purpose of CLL. Unfortunately, no one thought to do such a thing and so the rules 
are taught to students by supervisors in a one-on-one setting. The first aim of this report is to present 
the rules of classical linear logic such that someone who has no prior knowledge can come away knowing 
enough to approach more literature on the topic without feeling out of depth. We will then provide a 
translation from the simply typed lambda calculus into classical linear logic which has been considered 
by many prominent figures, but never been written out in full. \\

\noindent
There are two forms of logic: classical logic, and constructive logic (also called intuitionistic logic). 
Classical logic is the foundation for classical mathematics and has a few defining features including 
logical operations, structural rules, and double negation. Logical operations connect two or more statements 
such as $+, \otimes$, etc. Structural rules are operations performed directly on judgements or sequents such 
as weakening and exchange. Double negation is the assertation that if a proposition P is true then the inverse 
of it's inverse is also true, $P = \neg \neg P$. Statements in classical logic can be either 
true or false, and cannot be undetermined. More specifically, if we have some proposition P, either P is 
true, or the negation of P is true. This means we can use proof by contradiction to show that a proposition 
is true if its negation is false. Put simply, in classical logic, a statement is true if it is not false. \\

\noindent
In constructive logic, we have a stronger notion of truth. We only accept a proposition as true if it has a 
constructive proof, so in order to prove an object exists we must either provide an example of such an object or 
explain how we would create it. Dually, we only accept a proposition to be false if we have a refutation of it in 
constructive logic. There is no expectation that a proposition be either true or false, it can simply be a proposition. 
Constructive logic is a logic of positive evidence which makes it more expressive than classical logic. Any 
proof of a proposition in classical logic can be translated into a proof of a weaker (although calssically 
equivalent) proposition in constructive logic. \\

\noindent
Our main topic is classical linear logic which was first presented by Girard as an extension of constructive logic 
with a classical framework. CLL is a substructural logic which makes use of double negation from classical logic, and 
constructivism from intuitionistic logic. The addition of exponentials allows us to use structural rules to some extent. 
It has a number of applications in computer science, namely in reasoning about resources, resource useage, ownership, 
and communication. The latter is seen under the Curry-Howard isomorphism and is the aspect we will be focusing on. \\

\noindent
The Curry-Howard isomorphism is a correspondence between proof systems and type systems where we see propositions as 
types, proofs as programs, and normalisation as computation. Here we use a variant of this found by Caires and Pfenning 
\cite{caires2010} with 
\begin{align*}
    \text{propositions }& \text{as session types,} \\
    \text{proofs } & \text{as processes,} \\
    \text{cut elimination } & \text{as computation.} 
\end{align*}

\noindent
This variant equates classical linear logic with a process calculus similar to the $\pi$-calculus and allows 
us to model classical linear logic as a parallel programming language. For this report we will use Wadler's 
\cite{wadler2014} variant on the $\pi$-calculus which he combines with CLL to create a session-typed process calculus, 
classical processes (CP). Wadler's variant extends the $\pi$-calculus so it has processes corresponding to 
each of the typing judgements needed for CLL. For each logical proposition we have a session type which 
describes a type for the communicating protocols. Session types  allow us to describe how two processes 
communicate along a channel. They are most often used in relation to the $\pi$-calculus which was introduced 
by Milner \cite{milner1992} as a way to model processes which communicate concurrently. \\

\noindent
When Girard first introduced classical linear logic \cite{girard1987}, he suggested that the multiplicative fragment 
on CLL has obvious links to parallel computation. 

% -----------------------------------------------------------------------------
\chapter{The Simply Typed Lambda Calculus}
\label{chap: stlc}

The lambda calculus was invented by Church the 1930s \cite{church1932} to provide a logical foundation for 
mathematics. However, his original system was soon proved to have some logical inconsistencies \cite{kleene1935}. 
Church then published the section of his system meant for dealing with functions in isolation 
which he proved to be consistent. This portion of his original work is now known as the untyped 
lambda calculus. Later on, he published the simply typed lambda calculus \cite{church1940} which is also logically 
consistent although computationally weaker than its untyped counterpart. Although his original 
system was designed as a foundation for mathematics, it was largely overlooked by the mathematic 
community who preferred set theory. It wasn't until the 1960s that the $\lambda$-calculus was 
discovered to be a useful tool in computer science. \\

\noindent
The simply typed lambda calculus ($\lambda$-calculus) is regarded as the most basic functional programming language. 
It can be used both as a simple programming language with which we can execute computation, and 
also as a mathematical object which we can perform proofs upon. Both of these aspects are 
extremely useful to computer scientists. Modelling computation helps us to understand how a 
language works, and viewing a language as a mathematical object allows us to prove that it 
is safe. 

\section{Statics}

\begin{figure}[h]
    \begin{align*}
        Types \quad \tau \quad ::= \quad & 1 &\text{unit} \\
        & \tau_1 + \tau_2 &\text{sum} \\
        & \tau_1 \times \tau_2 &\text{product} \\
        & \tau_1 \rightarrow \tau_2 &\text{function} \\
        \\
        Terms \quad e \quad ::= \quad & x &\text{variable} \\
        & \LamInl{e} &\text{left injection} \\
        & \LamInr{e} &\text{right injection} \\
        & \LamCase{e}{x}{e_1}{y}{e_2} &\text{case} \\
        & \Tuple{e_1}{e_2} &\text{pair} \\
        & \Proj{e} &\text{left projection} \\
        & \Proj[2]{e} &\text{right projection} \\
        & \Lam{x}[\tau]{e} &\text{abstraction} \\
        & e_1 (e_2) &\text{application}
    \end{align*}
    \caption{Types and Terms for the $\lambda$-Calculus}
    \label{fig: tt stlc}
\end{figure}

\noindent
The simply typed lambda calculus is a language for working with functions. If we have some typing
relation $\Gamma \vdash x: \sigma$ this means that in the context $\Gamma$, $x$ has type $\sigma$,
and $x$ is said to be well-typed. We can see from \ref{fig: tt stlc} that we have three composite
types, and terms can take on specific forms, the rules for which are shown in \ref{fig: sr stlc}. The
sum type is used to represent disjoint union so if $\Gamma \vdash x: \tau_1 + \tau_2$ then the type 
of $x$ is either $\tau_1$ or $\tau_2$. The product type is used to represent pairs, so if we have
$\Gamma \vdash \langle e_1, e_2 \rangle : \tau_1 \times \tau_2$ then we know that we have both 
$e_1: \tau_1$ and $e_2: \tau_2$. Finally, the function type is of course used to represent functions
so if we have $\Lam{x}[\sigma]{e}: \sigma \rightarrow \tau$ then this is equivalent to having some 
function $f : \sigma \rightarrow \tau$ so we must have that $x: \sigma$ and $f(x) = e: \tau$. 
So our function $f$ takes in some parameter of type $\sigma$ and outputs a result with type $\tau$.\\

\noindent
The rules in \ref{fig: sr stlc} have premises on the top and conclusions on the bottom. So for any rule
we know that if we have everything above the line, then we can conclude anything below the line. Rules 
without any premises are called axioms and require no justification.
We take a constructor to be a rule which can create something of a certain type, and a destructor to be 
a rule which reduces something of that type to a simpler type. \\


\begin{figure}[h]
    \begin{mathpar}
        \inferH{Var}{
        }{
            \Gamma, \DeclVar{x}{\sigma} \vdash \DeclVar{x}{\sigma}
        }\qquad
        \inferH{Unit}{
        }{
            \Gamma \vdash \UnitV : 1
        }\\
        \inferH{In-l}{
            \Gamma \vdash e: \tau_1
        }{
            \Gamma \vdash \LamInl{e}: \tau_1 + \tau_2
        }\qquad
        \inferH{In-r}{
            \Gamma \vdash e: \tau_2
        }{
            \Gamma \vdash \LamInr{e}: \tau_1 + \tau_2
        }\\
        \inferH{Case}{
            \Gamma \vdash e: \tau_1 + \tau_2 \\
            \Gamma, x: \tau_1 \vdash e_1: \tau \\
            \Gamma, y: \tau_2 \vdash e_2: \tau
        }{
            \Gamma \vdash \LamCase{e}{x}{e_1}{y}{e_2}: \tau
        }\\
        \inferH{Pair}{
            \Gamma \vdash e_1: \tau_1 \\
            \Gamma \vdash e_2: \tau_2
        }{
            \Gamma \vdash \Tuple{e_1}{e_2} : \tau_1 \times \tau_2
        }\qquad
        \inferH{Prj-l}{
            \Gamma \vdash e: \tau_1 \times \tau_2
        }{
            \Gamma \vdash \Proj{e}: \tau_1
        }\qquad
        \inferH{Prj-r}{
            \Gamma \vdash e: \tau_1 \times \tau_2
        }{
            \Gamma \vdash \Proj[2]{e}: \tau_2
        }\\
        \inferH{Abs}{
            \Gamma, x: \sigma \vdash e: \tau
        }{
            \Gamma \vdash \Lam{x}[\sigma]{e}: \sigma \rightarrow \tau
        }\qquad
        \inferH{App}{
            \Gamma \vdash e_1: \sigma \rightarrow \tau \\
            \Gamma \vdash e_2: \sigma
        }{
            \Gamma \vdash e_1 (e_2): \tau
        }\\
    \end{mathpar}
    \caption{Static Rules for the $\lambda$-Calculus}
    \label{fig: sr stlc}
\end{figure}

\noindent
The \ruleref{Var} rule simply states that if we have $x: \sigma$ in some environment $\Gamma$, 
then we know $x$ has type $\sigma$. The \ruleref{Unit} rule states that if we have an empty pair, then this has 
the unit type, 1. \ruleref{In-r} is a constructor for the sum type which allows us to use injection 
to derive that $\LamInl{e}$ has type $\tau_1 + \tau_2$ given that $e$ has type $\tau_1$, and similarly for 
\ruleref{In-l}. The \ruleref{Case} rule is the destructor for the sum type which branches on either
side depending on the specific instance given. %???? strange explanation%
The \ruleref{Pair} rule allows us to construct a product type $\tau_1 \times \tau_2$ given two terms 
with each of those types. \ruleref{Prj-l} destructs some pair with a product type $\tau_1 \times \tau_2$ 
into a term with type $\tau_1$, and similarly \ruleref{Prj-r} will destruct the pair into a term 
with type $\tau_2$. 
The \ruleref{Abs} rule refers to lambda abstraction and is a constructor for the product type. It allows
us to create a function $\Lam{x}[\sigma]{e} \sigma \rightarrow \tau$ if we can obtain some $e$ of type $\tau$
from $\Gamma$ and $ x: \sigma$. The \ruleref{App} rule is the destructor of the function type, function 
application. If we have some function $e_1: \sigma \rightarrow \tau$ and some variable $e_2$ of type 
$\sigma$ then we can apply the function $e_1$ to $e_2$ and obtain a result with type $\tau$.
\\

\section{Dynamics}

\noindent
We know how the typing system works for the $\lambda$-calculus, but we also need to understand 
it's computational behaviour, i.e. the dynamics of a program. To understand these rules fully, we 
will need the substituition lemma. \\

\noindent
\textbf{Substituition:} If we have $\Gamma \vdash e: \tau$ and $\Gamma, x: \tau \vdash u: \sigma$, 
then $\Gamma \vdash u[e/x]: \sigma$. 

\begin{figure}
    \vspace*{-1em}
    \begin{align*}
        x[e/x] &\EqDef e &w[e/x] &\EqDef w \\\\
        \LamInl{u}[e/x] &\EqDef \LamInl{u[e/x]} &\LamInr{u}[e/x] &\EqDef \LamInr{u[e/x]} \\
        &\qquad \LamCase{u}{y}{e_1}{z}{e_2}[e/x] &\EqDef \LamCase{u[e/x]}{y}{e_1[e/x]}{z}{e_2[e/x]} \\\\
        \UnitV[e/x] &\EqDef \UnitV &\Tuple{e_1}{e_2}[e/x] &\EqDef \Tuple{e_1[e/x]}{e_2[e/x]} \\
        \Proj{u}[e/x] &\EqDef \Proj{u[e/x]} &\Proj[2]{u}[e/x] &\EqDef \Proj[2]{u[e/x]} \\\\
        (\Lam{w}{u})[e/x] &\EqDef \Lam{w}{u[e/x]} &(e_1 (e_2))[e/x] &\EqDef (e_1[e/x])(e_2[e/x]) \\      
    \end{align*}
    \caption{Substitution for the $\lambda$-Calculus}
    \label{fig: dr stlc}
\end{figure}

\noindent
Now we understand how substituition works, we can fully appreciate the dynamic rules 
for the $\lambda$-calculus which are shown below.

\begin{figure}[h]
    \begin{mathpar}
        \inferH{Val-Unit}{
        }{
            \UnitV \ \text{val}
        }\qquad
        \inferH{Val-In-l}{
        }{
            \LamInl{e} \ \text{val}
        }\qquad
        \inferH{Val-In-r}{
        }{
            \LamInr{e} \ \text{val}
        }\qquad
        \inferH{Val-Pair}{
        }{
            \Tuple{e_1}{e_2} \ \text{val}
        }\qquad
        \inferH{Val-Abs}{
        }{
            \Lam{x}[\sigma]{e} \ \text{val}
        }\\
        \inferH{D-Case-In-l}{
        }{
            \LamCase{\LamInl{e}}{x}{e_1}{y}{e_2} \Step e_1[e/x]
        }\qquad
        \inferH{D-Case-In-r}{
        }{
            \LamCase{\LamInr{e}}{x}{e_1}{y}{e_2} \Step e_2[e/y]
        }\\
        \inferH{D-Case}{
            e \Step e'
        }{
            \LamCase{e}{x}{e_1}{y}{e_2} \Step \LamCase{e'}{x}{e_1}{y}{e_2}
        }\\
        \inferH{D-Prj-Pair-l}{
        }{
            \Proj{\Tuple{e_1}{e_2}} \Step e_1
        }\qquad
        \inferH{D-Prj-Pair-r}{
        }{
            \Proj[2]{\Tuple{e_1}{e_2}} \Step e_2
        }\\
        \inferH{D-Prj-l}{
            e \Step e'
        }{
            \Proj{e} \Step \Proj{e'}
        }\qquad
        \inferH{D-Prj-r}{
            e \Step e'
        }{
            \Proj[2]{e} \Step \Proj[2]{e'}
        }\\
        \inferH{D-Beta}{
        }{
            (\Lam{x}[\sigma]{e_1})(e_2) \Step e_1[e_2/x]
        }\qquad
        \inferH{D-App}{
            e_1 \Step e'_1
        }{
            e_1(e_2) \Step e'_1(e_2)
        }
    \end{mathpar}
    \caption{Dynamic Rules for the $\lambda$-Calculus}
    \label{fig: dr stlc}
\end{figure}

\noindent
Again we have premises above the line and conclusions below the line. However, here 
we make use of $e_1 \Step e_2$ to signify a transition from one term to another. This 
transition can be thought of as a change of state from $e_1$ to $e_2$. Clearly, any 
``val'' rule simply states that terms of the given structure are values and thus cannot 
transition. Apart from these val rules, we have two distinct transition types: instruction 
transitions which perform computation and require no premises, and search transitions 
which allow for subterm computation. Together these determine the order of evaluation of 
terms. If we can perform a search transition on some term then we must do so before performing 
an instruction transition. For each instruction transition, there is an implication that 
none of the subterms can transition to another state while the whole term is in it's current 
state. \\

\noindent
The \ruleref{D-Case-In-l} rule substitutes $e$ for $x$ in $e_1$ if $\LamInl{e}$ is the first subterm, 
and \ruleref{D-Case-In-r} performs the same substitution in $e_2$ if we have $\LamInr{e}$ instead. 
The \ruleref{D-Case} rule transitions from a case statement with $e$ to the same case statement 
but with $e'$ if $e$ can transition to $e'$. \ruleref{D-Prj-Pair-l} offers the left projection of 
a pair, and similarly \ruleref{D-Prj-Pair-r} offers the right projection. \ruleref{D-Prj-l} and 
\ruleref{D-Prj-r} transition from the projection of $e$ to the projection of $e'$ if 
$e \Step e'$. The \ruleref{D-Beta} rule performs a beta reduction on a lambda term and replaces $x$ 
with $e_2$ in $e_1$. Finally, \ruleref{D-App} transitions from the application $e_1 (e_2)$ to 
$e'_1 (e_2)$ if $e_1 \Step e'_1$. \\

\noindent
Combining both the dynamics and statics, we can ensure that the $\lambda$-calculus is type safe 
if the following properties hold: \\

\noindent
%\textbf{Finality:} If $e$ val then there is no $e'$ where $e \Step e'$. \\\\
%\textbf{Determinism:} If $e \Step e_1$ and $e \Step e_2$ then $e_1 \equiv e_2$. \\\\
\textbf{Progress:} If we have some term $e$ where $\vdash \DeclVar{e}{\tau}$ either $e$ is a 
value, or $e \Step e'$ for some $e'$. \\\\
\textbf{Preservation:} If we have $\vdash \DeclVar{e}{\tau}$ and $e \Step e'$ we must have 
$\vdash \DeclVar{e'}{\tau}$. \\

\noindent
Progress states that computation will continue until evaluation is complete, i.e. until we have a 
value. Preservation tells us that types are preserved under evaluation. Together they show that 
well-typed programs don't go wrong. 

%proof of progress and preservation for stlc?

% -----------------------------------------------------------------------------

\chapter{Classical Linear Logic}
\label{chap:execution}

Classical linear logic (CLL) is a sequent calculus which differs from both classical and intuitionistic 
logic in a number of ways. 
In classical linear logic variables are denoted by capital letters $A, B, ...$ and the duals to these 
variables are denoted $\negg{A}, \negg{B}, ...$ where $\negg{A}$ is the negation of A such that 
$\negg{\negg{A}} = A$. This negation is not present in intuitionistic linear logic, but is a large part 
of classical linear logic and all propositions have a dual. \\

\noindent
The next difference is that we replace the usual two-sided sequent with a one-sided sequent, so a sequent 
in classical logic such as:
\begin{mathpar}
  A_1, A_2, ...,A_n \vdash B_1, B_2, ..., B_m
\end{mathpar}
would read as
\begin{mathpar}
  \vdash \negg{A_1}, \negg{A_2}, ..., \negg{A_n}, B_1, B_1, ..., B_m
\end{mathpar}
in classical linear logic.
This makes CLL slightly easier to read as we don't have to worry about propositions on the left-hand 
side of our sequent. We can see the use linear negation in our new one-sided sequent. \\

\noindent
We also have that any proposition must be used exactly as many times as it appears. We cannot have some 
proposition $A$ and not make use of it, and similarly we cannot use it any more than once if it only appears 
once. \\

\noindent
Our propositions for CLL are as follows:

\begin{figure}[h]
  \begin{align*}
      & X &\text{variable} \\
      & \negg{X} &\text{variable dual} \\
      & A \otimes B &\text{tensor} \\
      & A \parr B &\text{par} \\
      & A \oplus B &\text{plus} \\
      & A \with B &\text{with} \\
      & \ofc A &\text{of course} \\
      & \whynot A &\text{why not} \\
      & 1 &\text{tensor unit} \\
      & \bot &\text{par unit} \\
      & 0 &\text{plus unit} \\
      & \top &\text{with unit} \\
  \end{align*}
  \caption{Propositions for Classical Linear Logic}
  \label{fig: p cll}
\end{figure}

\noindent
Generally when researching the basics of classical linear logic, one may find the vending machine example. It is 
widely used to describe the propositions of CLL in a way that is easy to understand. Say we have a vending machine 
which takes \pounds 1 coins and has the options of Tea or Coffee. If we see the tensor symbol on the machine: \emph{Tea} 
$\otimes$ \emph{Coffee}, then inserting \pounds 1 will give us both Tea and Coffee. However if we see \emph{Tea} $\with$ \emph{Coffee} then 
inserting \pounds 1 will give us a choice of either Tea or Coffee. If you can't decide what drink you want then finding a 
vending machine which says \emph{Tea} $\oplus$ \emph{Coffee} will take our \pounds 1 and give us either Tea or Coffee making the decision 
for us. Par is not quite so easy to describe, if we stay with our vending machines then we would most likely see 
$\pounds 1 \ \parr$ \emph{Tea} which would imply that inserting \pounds 1 into the vending machine will then give us Tea. 
Having a term $A \parr B$ in classical linear logic is equivalent to having a term $\negg{A} \multimap B$ in linear logic:
``not A linearly implies B''. \\

\noindent
The exponential fragment also fits in with this slightly differently. If we have a term $\ofc \pounds 1$ this would indicate 
that we have multiple \pounds 1 coins which we may use at the vending machine. We may also see a term $\whynot \pounds 1$ 
on any of these vending machines which would indicate that it is possible to use the machine more than once by inserting 
multiple \pounds 1 coins. Both of course ($\ofc$) and why not ($\whynot$) mean that whatever follows it can be used any 
number of times, or even never used at all. \\

\noindent
For this report, we will modify these definitions slightly so they fit with our process calculus. \\

\noindent
Tensor, $A \otimes B$ represents multiplicative conjunction and means output A, then continue as B. The dual to tensor 
is par which represents multiplicative disjunction. We would read $A \parr B$ as input A, then continue as B. \\

\noindent
We have plus for additive disjunction so $A \oplus B$ means select either A or B.  Dual to this we have with: 
$A \with B$ meaning offer a choice between A and B. \\

\noindent
Our exponential component consists of of course and why not, where 
$\ofc A$ means we have a server which can accept many copies of A, and $\whynot A$ means we have a client 
who may request many copies of A. \\

\noindent
It is important to note that every proposition has a dual, specifically:

\begin{align*}
  \negg{(A \otimes B)} &= \negg{A} \parr \negg{B} & \negg{(A \parr B)} &= \negg{A} \otimes \negg{B} \\
  \negg{(A \oplus B)} &= \negg{A} \with \negg{B} & \negg{(A \with B)} &= \negg{A} \oplus \negg{B}\\
  \negg{(\ofc A)} &= \whynot \negg{A} & \negg{(\whynot A)} &= \ofc \negg{A}
\end{align*}

\noindent
Classical linear logic naturally lends itself to parallelism, so we present the rules of CLL 
alongside a process calculus similar to $\pi$ calculus. The processes for which are as follows:

\begin{figure}[h]
  \begin{align*}
    \Proc{P} ::= & \\
    & \Proc{P \mathbin{|} Q} & \text{parallel composition} \\
    & \Proc{\Link{x}{y}} & \text{link $\Name{x}$ with $\Name{y}$} \\
    & \Proc{\Out{y}{x}{P}} & \text{output $\Name{x}$ on channel $\Name{y}$} \\
    & \Proc{\In{y}{x}{P}} & \text{input $\Name{x}$ on channel $\Name{y}$} \\
    & \Proc{\Inl{x}{P}} & \text{left selection} \\
    & \Proc{\Inr{x}{P}} & \text{right selection} \\
    & \Proc{\Casep{x}{P}{Q}} & \text{choice} \\
    & \Proc{\New*{x}{}{P \mathbin{|} Q}} & \text{connect on channel $\Name{x}$} \\
    & \Proc{\whynot \Out{y}{x}{P}} & \text{client request} \\
    & \Proc{\ofc \In{y}{x}{P}} & \text{server accept} \\
    & \Proc{\Out{y}{\ }{P}} & \text{empty output} \\
    & \Proc{\In{y}{\ }{P}} & \text{empty input} \\
    & \Proc{\EmCase{x}} & \text{empty choice} \\
    & \Proc{P[\Name{x} / \Name{y}]} & \text{substitution}
  \end{align*}
  \caption{Process Calculus for CLL}
  \label{fig: p cp}
\end{figure}

\noindent
A process $\Proc{P}$ can take on a few forms. Link says that we can connect two channels in 
such a way that any message delivered to $\Name{y}$ will be forwarded over $\Name{x}$ and vice verca.
For an output such as $\Proc{\Out{y}{x}{(P \mathbin{|} Q)}}$ we have that $\Name{y}$ is bound in 
$\Proc{P}$, but is not bound in $\Proc{Q}$. For the input $\Proc{\In{y}{x}{P}}$, $\Name{y}$ is bound 
in $\Proc{P}$. In $\Proc{\New*{x}{}{P \mathbin{|} Q}}$, $\Name{x}$ is bound in both $\Proc{P}$ and 
$\Proc{Q}$. Server accept is just like input so for $\Proc{\In{\ofc y}{x}{P}}$, we would have that 
$\Name{x}$ is bound in $\Proc{P}$, and similarly for $\Proc{\Out{\whynot y}{x}{P}}$. 

\section{Rules for CLL}

\noindent
We can now combine our process calculus with our propositions for CLL and present the rules for 
classical linear logic alongside their corresponding processes. Our typing judgements have the form 
$\IsProc{P}{\Gamma, \Name{x}: A}$ where \Proc{P} is a process, $\Gamma$ is a type environment, 
$\Name{x}$ is a channel, and $A$ is a CLL proposition. This typing judgement means we have some process 
\Proc{P} communicating along channel $\Name{x}$ obeying protocol $A$. Processes can communicate over 
multiple channels following different protocols, and as we saw in \ref{fig: p cp} the processes 
may also take on varying forms.

\begin{figure}[!]
  \begin{mathpar}
    \inferH{Axiom}{
    }{
      \IsProc{\Link{x}{y}}{\Name{x} : \negg{A}, \Name{y} : A}
    }\qquad
    \inferH{Cut}{
      \IsProc{P}{\Gamma, \Name{x} : \negg{A}}\\
      \IsProc{Q}{\Gamma, \Name{x} : {A}}   
    }{
      \IsProc{\New*{x}{}{P \mathbin{|} Q}}{\Gamma, \Delta}
    }\\

    \inferH{Tensor}{
        \IsProc{P}{\Gamma, \Name{x} : A} \\
        \IsProc{Q}{\Delta, \Name{y} : B}
    }{
        \IsProc{\Out{y}{x}{(P \mathbin{|} Q)}}{\Gamma, \Delta, \Name{y} : A \otimes B}
    } \qquad
    \inferH{Par}{
          \IsProc{P}{\Gamma, \Name{x} : A, \Name{y} : B} 
    }{
          \IsProc{\In{y}{x}{P}}{\Gamma, \Name{y} : A \parr B}
    }\\

    \inferH{Plus-L}{
      \IsProc{P}{\Gamma, \Name{x} : A}
    }{
      \IsProc{\Inl{x}{P}}{\Gamma, \Name{x} : A \oplus B}
    } \qquad
    \inferH{Plus-R}{
      \IsProc{P}{\Gamma, \Name{x} : B}
    }{
      \IsProc{\Inr{x}{P}}{\Gamma, \Name{x} : A \oplus B}
    }
    \qquad
    \inferH{With}{
      \IsProc{P}{\Gamma, \Name{x} : A} \\
      \IsProc{Q}{\Gamma, \Name{x} : B}
    }{
      \IsProc{\Casep{x}{P}{Q}}{\Gamma, \Name{x} : A \with B}
    }\\

    \inferH{Weakening}{
      \IsProc{P}{\Gamma}
    }{
      \IsProc{P}{\Gamma, \Name{x} : \whynot A}
    }\qquad
    \inferH{Contraction}{
      \IsProc{P}{\Gamma, \Name{x} : \whynot A, \Name{y} : \whynot A}
    }{
      \IsProc{P[\Name{x} / \Name{y}]}{\Gamma, \Name{x} : \whynot A}
    }\qquad
    \inferH{Dereliction}{
      \IsProc{P}{\Gamma, \Name{x} : A}
    }{
      \IsProc{\whynot \Out{y}{x}{P}}{\Gamma, \Name{y} : \whynot A}
    }\\

    \inferH{Promotion}{
      \IsProc{P}{\whynot \Gamma, \Name{x} : A}
    }{
      \IsProc{\ofc \In{y}{x}{P}}{\whynot \Gamma, \Name{y} : \ofc A}
    }\\

    \inferH{}{
    }{
      \IsProc{\Out{x}{\ }{P}}{\Name{x}: 1}
    }\qquad
    \inferH{}{
      \IsProc{P}{\Gamma}
    }{
      \IsProc{\In{x}{\ }{P}}{\Gamma, \Name{x}: \bot}
    }\qquad
    \inferH{}{
    }{
      \IsProc{\EmCase{x}}{\Gamma, \Name{x}: \top}
    }
  \end{mathpar}
  \caption{Rules for Classical Linear Logic}
  \label{fig: r cll cp}
\end{figure}

\noindent
The \ruleref{Axiom} states that if we have some variable A, then we also have its dual $\negg{A}$. 
For the process calculus, we see that if we have two channels following dual protocols then any 
input along $\Name{x}$ is sent as output along $\Name{y}$.
The \ruleref{Cut} rule allows us to connect two processes together. As the protocols are dual, 
any transmissions and selections over one correspond with receives and choices over the other. This 
along with communication only via one channel ensures that the processes cannot get stuck. \\

\noindent
The \ruleref{Tensor} rule outputs a fresh channel x along y, then continues as P and Q in parallel.
As P and Q communicate over different channels, we have disjoint concurrency so the processes cannot 
communicate with each other. The new process, $\Proc{P \mathbin{|} Q}$ communicates over channel $\Name{y}$.
The rule \ruleref{Par} inputs A, then continues as B. This is known as connected concurrency as P can 
communicate along both $\Name{x}$ and $\Name{y}$. \\

\noindent
The rule \ruleref{Plus-L} indicates left selection, and similarly \ruleref{Plus-R} indicates right 
selection. The process $\Proc{\Inl{x}{P}}$ obeys protocol $A \oplus B$ by requesting the left option 
from a choice sent along $x$. The process for $\Proc{\Inr{x}{P}}$ is symmetric. 
The \ruleref{With} rule offers a choice between processes $\Proc{P}$ and $\Proc{Q}$. The new process 
$\Proc{\Casep{x}{P}{Q}}$ will receive a selection over channel $x$ and execute either $\Proc{P}$ or 
$\Proc{Q}$ accordingly. \\

\noindent
\ruleref{Weakening} lets us consider a process which doesn't communicate or follow a protocol to be 
a process which communicates along a channel $\Name{x}$ with protocol \whynot A. 
If a process $\Proc{P}$ communicates along two channels, both following the same protocol \whynot A, 
then we can use \ruleref{Contraction} to substitute one channel for another so $\Proc{P}$ communicates 
along only one channel following protocol \whynot A.
\ruleref{Dereliction} allows us to.

\section{Cut Reduction}

\noindent 
Just like in the simply typed lambda calculus, there are dynamic rules for classical linear logic. 
We have already seen the \ruleref{Cut} rule and the dynamics for CLL make use of this rule to simplify 
terms via cut reduction. 

\begin{mathpar}

  \inferH{}{
    \inferH{}{
    }{
      \IsProc{\Link{x}{y}}{\Name{x}: \negg{A}, \Name{y}: A}
    }\\
    \inferH{}{
    }{
      \IsProc{P}{\Gamma, \Name{x}: A}
    }
  }{
    \IsProc{\New*{x}{}{\Link{x}{y} \mathbin{|} P}}{\Gamma, \Name{y}: A}
  } \quad \Longrightarrow \quad
  \IsProc{P[\Name{y} / \Name{x}]}{\Gamma, \Name{y}: A} \\\\

  \inferH{}{
    \inferH{}{
      \IsProc{P}{\Gamma, \Name{x} : \negg{A}} \\
      \IsProc{Q}{\Delta, \Name{y} : \negg{B}}
    }{
      \IsProc{\Out{y}{x}{(P \mathbin{|} Q)}}{\Gamma, \Delta, \Name{y} : \negg{A} \otimes \negg{B}} \\
    }\\
    \inferH{}{
      \IsProc{R}{\Theta, \Name{x} : A, \Name{y} : B}
    }{
      \IsProc{\In{y}{x}{R}}{\Theta, \Name{y} : A \parr B}
    }
  }{
    \IsProc{\New*{y}{}{\Out{y}{x}{(P \mathbin{|} Q)} \mathbin{|} \In{y}{x}{R}}}{\Gamma, \Delta, \Theta}
  } \quad \Longrightarrow \\

  \inferH{}{
    \inferH{}{
      \IsProc{P}{\Gamma, \Name{x} : \negg{A}} \\
      \IsProc{R}{\Theta, \Name{x} : A, \Name{y}: B}
    }{
      \IsProc{\New*{x}{}{P \mathbin{|} Q}}{\Gamma, \Theta, \Name{y} : B} \\
    }\\
    \inferH{}{
    }{
      \IsProc{Q}{\Delta, \Name{y} : \negg{B}}
    }
  }{
    \IsProc{\New*{y}{}{\New*{x}{}{P \mathbin{|} R} \mathbin{|} Q}}{\Gamma, \Delta, \Theta}
  }\\\\

  \inferH{}{
    \inferH{}{
      \IsProc{P}{\Gamma, \Name{x}: A}
    }{
      \IsProc{\Inl{x}{P}}{\Gamma, \Name{x}: A \oplus B}
    }\\
    \inferH{}{
      \IsProc{Q}{\Delta, \Name{x}: \negg{A}}\\
      \IsProc{R}{\Theta, \Name{x}: \negg{B}}
    }{
      \IsProc{\Casep{x}{Q}{R}}{\Delta, \Name{x}: A \with B}
    }
  }{
    \IsProc{\New*{x}{}{\Inl{x}{P} \mathbin{|} \Casep{x}{Q}{R}}}{\Gamma, \Delta}
  } \quad \Longrightarrow \quad
  \inferH{}{
    \IsProc{P}{\Gamma, \Name{x}: A}\\
    \IsProc{Q}{\Delta, \Name{x}: \negg{A}}
  }{
    \IsProc{\New*{x}{}{P \mathbin{|} Q}}{\Gamma, \Delta}
  }\\\\

  \inferH{}{
    \inferH{}{
      \IsProc{P}{\whynot \Gamma, \Name{x}: A}
    }{
      \IsProc{\ofc \In{y}{x}{P}}{\whynot \Gamma, \Name{y}: \ofc A}
    }\\
    \inferH{}{
      \IsProc{Q}{\Delta, \Name{x}: \negg{A}}
    }{
      \IsProc{\whynot \Out{y}{x}{Q}}{\Delta, \Name{y}: \negg{\whynot A}}
    }
  }{
    \IsProc{\New*{y}{}{\ofc \In{y}{x}{P} \mathbin{|} \whynot \Out{y}{x}{Q}}}{\whynot \Gamma, \Delta}
  } \quad \Longrightarrow \quad
  \inferH{}{
    \IsProc{P}{\whynot \Gamma, \Name{x}: A}\\
    \IsProc{Q}{\Delta, \Name{x}: \negg{A}}
  }{
    \IsProc{\New*{x}{}{P \mathbin{|} Q}}{\whynot \Gamma, \Delta}
  }\\\\

  \inferH{}{
    \inferH{}{
      \IsProc{P}{\whynot \Gamma, \Name{x}: A}
    }{
      \IsProc{\ofc \In{y}{x}{P}}{\whynot \Gamma, \Name{y}: \ofc A}
    }\\
    \inferH{}{
      \IsProc{Q}{\Delta}
    }{
      \IsProc{Q}{\Delta, \Name{x}: \negg{\whynot A}}
    }
  }{
    \IsProc{\New*{y}{}{\ofc \In{y}{x}{P} \mathbin{|} Q}}{\whynot \Gamma, \Delta}
  } \quad \Longrightarrow \quad
  \inferH{}{
    \IsProc{Q}{\Delta}
  }{
    \IsProc{Q}{\whynot \Gamma, \Delta}
  }\\\\

  \inferH{}{
    \inferH{}{
      \IsProc{P}{\whynot \Gamma, \Name{x}: A}
    }{
      \IsProc{\ofc \In{y}{x}{P}}{\whynot \Gamma, \Name{y}: \ofc A}
    }\\
    \inferH{}{
      \IsProc{Q}{\Delta, \Name{y}: \whynot A, \Name{z}: \whynot A}
    }{
      \IsProc{Q[\Name{y} / \Name{z}]}{\Delta, \Name{y}: \whynot A}
    }
  }{
    \IsProc{\New*{y}{}{\ofc \In{y}{x}{P} \mathbin{|} Q[\Name{y} / \Name{z}]}}{\whynot \Gamma, \Delta}
  }\quad \Longrightarrow \\

  \inferH{}{
    \inferH{}{
      \IsProc{P}{\whynot \Gamma, \Name{x}: A}
    }{
      \IsProc{\ofc \In{y}{x}{P}}{\whynot \Gamma, \Name{y}: \ofc A}
    }\\
    \inferH{}{
      \inferH{}{
        \IsProc{P'}{\whynot \Gamma', \Name{w}: A}
      }{
        \IsProc{\ofc \In{z}{w}{P}}{\whynot \Gamma', \Name{z}: \ofc A}
      }\\
      \IsProc{Q}{\Delta, \Name{y}: \negg{\whynot A}, \Name{z}: \negg{\whynot A}}
    }{
      \IsProc{\New*{z}{}{\ofc \In{z}{w}{P} \mathbin{|} Q}}{\whynot \Gamma', \Delta, \Name{x}: \negg{\whynot A}}
    }
  }{
    \inferH{}{
      \IsProc{\New*{y}{}{\ofc \In{y}{x}{P} \mathbin{|} \New*{z}{}{\ofc \In{z}{w}{P'} \mathbin{|} Q}}}{\whynot \Gamma, \whynot \Gamma', \Delta}
    }{
      \IsProc{\New*{y}{}{\ofc \In{y}{x}{P} \mathbin{|} \New*{z}{}{\ofc \In{z}{y}{P} \mathbin{|} Q}}}{\whynot \Gamma, \Delta}
    }
  }\\\\

  \inferH{}{
    \inferH{}{
      \IsProc{P}{\Gamma}
    }{
      \IsProc{\In{x}{\ }{P}}{\Gamma, \Name{x}: \bot}
    }\\
    \inferH{}{
    }{
      \IsProc{\Out{x}{\ }{Q}}{\Name{x}: 1}
    }
  }{
    \IsProc{\New*{x}{}{\ }\In{x}{\ }{P} \mathbin{|} \Out{x}{\ }{Q}}{\Gamma}
  }\quad \Longrightarrow \quad
  \IsProc{P}{\Gamma}
\end{mathpar}


% -----------------------------------------------------------------------------

\chapter{Translation}

Here we present a translation from the simply typed lambda calculus into classical linear logic. First 
we will give the direct translation of each type, then a detailed derivation of how we got there along 
with an explanation of what exactly is happening in our translated protocol. We 
make use of the $^{\circ}$ symbol to denote translation of a type. In some of the translations, we will 
also use the $^\dagger$ symbol which is simply a renaming of the translation and acts in the same way as 
$^\circ$. On the left hand side we have the lambda calculus type, and on the right we have its classical 
linear logic proposition. 
\begin{figure}[h]
  \begin{align*}
      (1)^\circ &= \top\\
      (A + B)^\circ &= \ofc A^\circ \oplus \ofc B^\circ \\
      (A \times B)^\circ &= A^\circ \with B^\circ \\
      (A \rightarrow B)^\circ &= \negg{(\ofc A^\circ)} \parr B^\circ
  \end{align*} 
  \caption{Translation of Types}
  \label{fig: tt}
\end{figure}

\noindent
The translations of 1 and $(A \times B)$ are as one would expect. The exponentials in the translation 
of $(A + B)$ indicate that we may select multiple times from an option we are sent. The translation 
of $(A \rightarrow B)$ makes use negation since we use some A to acquire some B. The exponential $\ofc A^\circ$ 
indicates that we can use our function multiple times, rather than just once. If we had just 
$\negg{A^\circ} \parr B^\circ$ on the CLL side, we would only be able to use the function once as we 
only have access to one $\negg{A^\circ}$. \\

\noindent
The translation of some environment $\Gamma$ is as follows: 
\begin{figure}[h]
  \begin{align*}
    \Gamma^\circ & \\
    (\cdot)^\circ &= \cdot \\
    (\Gamma, x:A)^\circ &= \Gamma^\circ, x: \negg{(\ofc A^\circ)} \\
    &= \Gamma^\circ, x: \whynot \negg{(A^\circ)}
  \end{align*}
  \caption{Translation of Contexts}
  \label{fig: tc} 
\end{figure}

\noindent
If we have a term $x:A$ from the lambda calculus, translates to $x: \negg{(\ofc A^\circ)}$. The negation 
occurs because contexts in the lambda calculus occur on the left hand side, but we only have right-handed 
sequents in classical linear logic. More interestingly, we have the exponentiation, $\ofc A^\circ$ rather than just $A^\circ$. 
This is because in the lambda calculus we may use a protocol as many times as we wish if it occurs in the 
context, but in classical linear logic we may only use something exactly as many times as it occurs. Hence, 
we have the exponentiation $(\ofc A^\circ)$ to indicate that we may use $A^\circ$ as many times as we desire. \\

\noindent
We will need a derivable rule, \ruleref{Depar} which takes a concurrent process $A \parr B$ communicating 
along a channel $z$ and disconnects them to give us both $A$ and $B$ on separate channels. We achieve this 
by creating copies of the individual components $x:A$ and $y:B$ along with their duals $w:\negg{A}, z:\negg{B}$, 
each of which communicate along separate channels and can forward information to their dual. We then output 
$w$ along $z$ allowing the protocols to continue in parallel. This gives us the dual, $z: \negg{A} \otimes \negg{B}$, 
to our original process $z:A \parr B$ and we can connect on channel $z$ to give us the disconnected processes $A$ 
and $B$.
\begin{mathpar}
  \begin{array}{ll}
    \inferH{}{
      \inferH{}{
        \IsProc{\Link{w}{x}}{\Name{x}: A, \Name{w}: \negg{A}}\\
        \IsProc{\Link{y}{z}}{\Name{y}: B, \Name{z}: \negg{B}}
      }{
        \IsProc{\Out{z}{w}{(\Link{w}{x} \mathbin{|} \Link{y}{z})}}{\Name{x}: A, \Name{y}: B, \Name{z}: \negg{A} \otimes \negg{B}}
      }\\
      \inferH{}{
      }{
        \IsProc{P}{\Gamma, \Name{z}: A \parr B}
      }
    }{
      \IsProc{\New*{z}{}{\Out{z}{w}{(\Link{w}{x} \mathbin{|} \Link{y}{z})} \mathbin{|} P}}{\Gamma, \Name{x}: A, \Name{y}: B}
    } &\quad \leadsto \\\\
    &\inferH{Depar}{
      \IsProc{P}{\Gamma, \Name{z}: A \parr B}
    }{
      \IsProc{\Depar{x}{y}{P}}{\Gamma, \Name{x}: A, \Name{y}: B}
    }
  \end{array}
\end{mathpar}

\begin{figure}
  \vspace{-2em}
  \begin{mathpar}
      %var
      {\Biggl\llbracket
      \inferH{T-Var}{
      }{
          \Gamma, x: A \vdash x: A
      }\Biggr\rrbracket}_\Name{y}
      \EqDef
      \inferH{}{
        \inferH{}{
        \IsProc{\Link{z}{y}}{\Name{z}: \negg{(A^\circ)}, \Name{y}: A^\circ}
        }{
          \IsProc{\whynot \Out{x}{z}{\Link{z}{y}}}{\Name{x}: \whynot \negg{(A^\circ)}, \Name{y}: A^\circ}
        }
      }{
        \IsProc{\whynot \Out{x}{z}{\Link{z}{y}}}{\Gamma^\circ, \Name{x}: \whynot \negg{(A^\circ)}, \Name{y}: A^\circ}
      }\\

      %unit
      {\Biggl\llbracket
      \inferH{T-Unit}{
      }{
      \Gamma \vdash \langle \rangle : 1
      }\Biggr\rrbracket}_\Name{x}
      \EqDef
      \inferH{}{
      }{
      \IsProc{\EmCase{x}}{\Gamma^{\circ}, \Name{x} : \top}
      }\\

      %left injection
      {\Biggl\llbracket
      \inferH{T-In-l}{
        \Gamma \vdash M:A
      }{
        \Gamma \vdash \LamInl{M}: A+B
      }\Biggr\rrbracket}_\Name{x}
      \EqDef
      \inferH{}{
        \inferH{}{
          \IsProc{\llbracket M \rrbracket_\Name{y}}{\Gamma^\circ, \Name{y}: A^\circ}
          }{
          \IsProc{\ofc \In{x}{y}{\llbracket M \rrbracket_\Name{y}}}{\Gamma^\circ, \Name{x}: \ofc A^\circ}
        }
      }{
        \IsProc{\Inl{x}{\ofc \In{x}{y}{\llbracket M \rrbracket_\Name{y}}}}{\Gamma^\circ, \Name{x}: \ofc A^\circ \oplus \ofc B^\circ}
      }
      \\

      %right injection
      {\Biggl\llbracket
      \inferH{T-In-r}{
        \Gamma \vdash M:B
      }{
        \Gamma \vdash \LamInr{M}: A+B
      }\Biggr\rrbracket}_\Name{x}
      \EqDef
      \inferH{}{
        \inferH{}{
          \IsProc{\llbracket M \rrbracket_\Name{y}}{\Gamma^\circ, \Name{y}: B^\circ}
          }{
          \IsProc{\ofc \In{x}{y}{\llbracket M \rrbracket_\Name{y}}}{\Gamma^\circ, \Name{x}: \ofc B^\circ}
        }
      }{
        \IsProc{\Inr{x}{\ofc \In{x}{y}{\llbracket M \rrbracket_\Name{y}}}}{\Gamma^\circ, \Name{x}: \ofc A^\circ \oplus \ofc B^\circ}
      }
      \\

      %case
      \Biggl\llbracket
      \inferH{T-Case}{
        \Gamma \vdash M: A+B \\
        \Gamma, x:A \vdash P:C \\
        \Gamma, x:B \vdash Q:C
      }{
        \Gamma \vdash \LamCase{M}{x}{P}{x}{Q}:C
      }\Biggr\rrbracket_\Name{y}
      \EqDef
      \inferH{}{
        \IsProc{\llbracket M \rrbracket_\Name{x}}{\Gamma^\circ, \Name{x}: \ofc A^\circ \oplus \ofc B^\circ} \\
        \IsProc{\llbracket P \rrbracket_\Name{y}}{\Gamma^\circ, \Name{y}: C^\circ, \Name{x}: \negg{(\ofc A^\circ)}} \\
        \IsProc{\llbracket Q \rrbracket_\Name{y}}{\Gamma^\circ, \Name{y}: C^\circ, \Name{x}: \negg{(\ofc B^\circ)}}
      }{
          \inferH{}{
          \IsProc{\llbracket M \rrbracket_\Name{x}}{\Gamma^\circ, \Name{x}: \ofc A^\circ \oplus \ofc B^\circ} \\
          \IsProc{\Casep{x}{\llbracket P \rrbracket_\Name{y}}{\llbracket Q \rrbracket_\Name{y}}}{\Gamma^\circ, \Name{y}: C^\circ, \Name{y}: C^\circ, \Name{x}: \negg{(\ofc A^\circ)} \with \negg{(\ofc B^\circ)}}
        }{
            \inferH{}{
            \IsProc{\New*{x}{}{\llbracket M \rrbracket_\Name{x} \mathbin{|} \Casep{x}{\llbracket P \rrbracket_\Name{y}}{\llbracket Q \rrbracket_\Name{y}}}}{\Gamma^\circ, \Name{y}: C^\circ, \Name{y}: C^\circ}
          }{
            \IsProc{\New*{x}{}{\llbracket M \rrbracket_\Name{x} \mathbin{|} \Casep{x}{\llbracket P \rrbracket_\Name{y}}{\llbracket Q \rrbracket_\Name{y}}}}{\Gamma^\circ, \Name{y}: C^\circ}
      }}}

      %product
      {\Biggl\llbracket
      \inferH{T-Prod}{
      \Gamma \vdash M : A \\
      \Gamma \vdash N : B
      }{
      \Gamma \vdash \Tuple{M}{N} : A \times B
      }\Biggr\rrbracket}_{\Name{x}}
      \EqDef
      \inferH{}{
      \IsProc{\llbracket M \rrbracket_\Name{x}}{\Gamma^{\circ}, \Name{x} : A^{\circ}} \\
      \IsProc{\llbracket N \rrbracket_\Name{x}}{\Gamma^{\circ}, \Name{x} : B^{\circ}}
      }{
      \IsProc{\Casep{x}{{\llbracket M \rrbracket}_\Name{x}}{{\llbracket N \rrbracket}_\Name{x}}}{\Gamma^{\circ}, \Name{x} : A^{\circ} \with B^{\circ}}
      }\\

      %left projection
      {\Biggl\llbracket
      \inferH{T-Prj-l}{
      \Gamma \vdash M : A \times B
      }{
      \Gamma \vdash \Proj{M} : A
      }\Biggr\rrbracket}_{\Name{x}}
      \EqDef
      \inferH{}{
      \inferH{}{
          \IsProc{\Link{x}{y}}{\Name{x} : A^\circ, \Name{y} : \negg{(A^{\circ})}}
      }{
      \IsProc{\Inl{y}{\Link{x}{y}}}{\Name{x} : A^{\circ}, \Name{y} : \negg{(A^{\circ})} \oplus \negg{(B^{\circ})}} \\
      }\\
      \inferH{}{
      }{
          \IsProc{{\llbracket M \rrbracket}_\Name{y}}{\Gamma^{\circ}, \Name{y} : A^{\circ} \with B^{\circ}}
      }}{
      \IsProc{\New*{y}{}{\Inl{y}{\Link{y}{x}} \mathbin{|} {\llbracket M \rrbracket}_\Name{y}}}{\Gamma^{\circ}, \Name{x} : A^{\circ}}
      }\\

      %right projection
      {\Biggl\llbracket
      \inferH{T-Prj-r}{
      \Gamma \vdash M : A \times B
      }{
      \Gamma \vdash \Proj[2]{M} : B
      }\Biggr\rrbracket}_{\Name{x}}
      \EqDef
      \inferH{}{
      \inferH{}{
          \IsProc{\Link{x}{y}}{\Name{x} : B^\circ, \Name{y} : \negg{(B^{\circ})}}
      }{
      \IsProc{\Inr{y}{\Link{x}{y}}}{\Name{x} : B^{\circ}, \Name{y} : \negg{(A^{\circ})} \oplus \negg{(B^{\circ})}} \\
      }\\
      \inferH{}{
      }{
          \IsProc{{\llbracket M \rrbracket}_\Name{y}}{\Gamma^{\circ}, \Name{y} : A^{\circ} \with B^{\circ}}
      }}{
      \IsProc{\New*{y}{}{\Inr{y}{\Link{y}{x}} \mathbin{|} {\llbracket M \rrbracket}_\Name{y}}}{\Gamma^{\circ}, \Name{x} : B^{\circ}}
      }\\

      %lambda abstraction
      {\Biggl\llbracket 
      \inferH{T-Lam}{
      \Gamma, x :A \vdash M :B
      }{
      \Gamma \vdash \Lam{x}[A]{M} : A \rightarrow B
      } \Biggr\rrbracket}_{\Name{y}}
      \EqDef
      \inferH{}{
      \IsProc{{\llbracket M \rrbracket}_\Name{y}}{\Gamma^{\circ}, \Name{x} : \negg{(\ofc A^{\circ})}, \Name{y} : B^{\circ}}
      }{
      \IsProc{\In{y}{x}{{\llbracket M \rrbracket}_\Name{y}}}{\Gamma^{\circ}, \Name{y} : \negg{(\ofc A^{\circ})} \parr B^{\circ}}
      }\\

      %application
      \Biggl\llbracket 
      \inferH{T-App}{
        \Gamma \vdash M:A \rightarrow B \\
        \Gamma \vdash N:A
      }{
        \Gamma \vdash MN:B
      }\Biggr\rrbracket_\Name{y}
      \EqDef
      \inferH{}{
        \inferH{}{
          \IsProc{\llbracket M \rrbracket_\Name{z}}{\Gamma^\circ, \Name{z}: \negg{(\ofc A^\circ)} \parr B^\circ}
        }{
          \IsProc{\Depar{x}{y}{\llbracket M \rrbracket_\Name{z}}}{\Gamma^\circ, \Name{x}: \negg{(\ofc A^\circ)}, \Name{y}: B^\circ}
        }\\
        \inferH{}{
          \IsProc{\llbracket N \rrbracket_\Name{u}}{\Gamma^\dagger, \Name{u}: A^\circ}
        }{
          \IsProc{\ofc \In{x^\dagger}{u}{\llbracket N \rrbracket_\Name{u}}}{\Gamma^\dagger, \Name{x^\dagger}: \ofc A^\circ}
        }
      }{
        \inferH{}{
          \IsProc{\New*{x}{}{\Depar{x}{y}{\llbracket M \rrbracket_\Name{z}} \mathbin{|} \ofc \In{x^\dagger}{u}{\llbracket N \rrbracket_\Name{u}}}}{\Gamma^\circ, \Gamma^\dagger, \Name{y}: B^\circ}
        }{
          \IsProc{\New*{x}{}{\Depar{x}{y}{\llbracket M \rrbracket_\Name{z}} \mathbin{|} \ofc \In{x}{u}{\llbracket N \rrbracket_\Name{u}}}}{\Gamma^\circ, \Name{y}: B^\circ}
        }}
  \end{mathpar}
  \caption{Translation from the $\lambda$-Calculus into Classical Processes}
  \label{fig: translatioin}
\end{figure}

\noindent
It is interesting to note that we have an inversion of polarity from the lambda calculus 
to classical linear logic. All type constructors in the lambda calculus become destructors 
in CLL and vice verca. \\ %why?? \\

\noindent
For our translation \ruleref{T-Var} with respect to $y$, we begin with the CLL axiom to  
give us two channels following dual protocols $A^\circ$ and $\negg{(A^\circ)}$ which 
share information with each other via forwarding. We then use 
dereliction to request $x$, and therefore the protocol $\whynot \negg{(A^\circ)}$ 
on a new channel $z$. Finally we use weakening to acquire an environment $\Gamma^\circ$ 
which we do not use. This gives us a typing judgement which is equivalent to the variable 
judgement in the lambda calculus. So, our final judgement says link 
$x: \negg{(A^\circ)}$ with $y: A^\circ$, then request $x: \negg{(A^\circ)}$ on a new channel $z$. \\

\noindent
For \ruleref{T-Unit} we simply use the empty case statement along with the unit for with. 
So, we create a channel $x$ with no protocols communicating along it. \\

\noindent
The translation for \ruleref{T-In-l} is fairly straightforward. We begin with a process communicating 
along channel $y$ following protocol $A^\circ$ and use promotion to input $y$ along channel $x$ to 
acquire the protocol $\ofc A^\circ$. We then use plus-l indicating we have selected the left choice of 
two protocols communicating on channel $x$ then execute the process $\llbracket M \rrbracket_x$. 
The translation of \ruleref{T-In-r} is equivalent, but we select the right choice. \\

\noindent
For the translation \ruleref{T-Case}, we use the anti-Barendregt convention and conveniently 
name variables in such a way that we do not have to rename them in the future. This is why 
we have $x: A$ and $x: B$ rather than the expected $y: B$. As in the 
lambda calculus, we start with three separate processes, each with the same environment. Our 
first step is to use the with rule to offer a choice between protocols $\negg{(A^\circ)}$ and 
$\negg{(B^\circ)}$ on channel $x$. Then we use the cut rule to connect the protocols on channel 
$x$. We then use contraction to remove copies of $y: C^\circ$ 
and allow our process to communicate along one channel. \\

\noindent
For \ruleref{T-Prod} we begin with two processes which communicate along the same channel $x$ 
following different protocols. Then we use the with rule to offer a choice between protocols 
$A^\circ$ and $B^\circ$ along the same channel $x$. \\

\noindent
The translation \ruleref{T-Prj-l} begins with the two dual protocols then requests the left 
option from a choice sent along $y$. We then use the cut rule to connect the dual protocols 
$\negg{A^\circ} \oplus \negg{B^\circ}$ and $A^\circ \with B^\circ$ on channel $y$ and acquire 
the left projection $A^\circ$ of the pair $A^\circ \with B^\circ$. The translation 
of \ruleref{T-Prj-r} is analogous. \\

\noindent
Our translation \ruleref{T-Lam} with respect to $y$ begins with a process following two protocols 
which commmunicate along serparate channels. Since we have $\Gamma, x:A \vdash$ on the $\lambda$-
calculus side, we translate this to $\negg{(\ofc A^\circ)}$. We then use the par rule to input 
$\negg{(\ofc A^\circ)}$ along channel $y$ then continue as $B^\circ$. The`of course' indicates 
that we can apply our function $\llbracket M \rrbracket_y$ to multiple inputs of the form 
$\negg{(A^\circ)}$. \\

\noindent
Finally, \ruleref{T-App} begins with a process following the protocol $\negg{\ofc A^\circ} \parr B^\circ$.
We use our derivable rule \ruleref{Depar} to disconnect the protocols components and allow them to 
communicate along separate channels. On the other side we have a process following protocol $A^\circ$ 
which communicate along channel $u$ with the environment $\Gamma^\dagger$ which is just a copy of 
the environment $\Gamma^\circ$. We use promotion to input $u$ along channel $x^\dagger$ (a copy of the 
channel $x$) to acquire the protocol $\ofc A^\circ$. We then connect 
the dual protocols $\negg{\ofc A^\circ}$ and $\ofc A^\circ$ on channel $x$ which gives us a process 
which communicate along channel $y$ following protocol $B^\circ$. Finally, we use contraction to merge 
$\Gamma^\circ$ and its copy $\Gamma^\dagger$ into a single environment. \\


\section{Proof of Simulation}

Now we have a translation from the $\lambda$-calculus into CLL, we want to ensure it is correct. 
We will do this using a proof of simulation to show that reduction in the $\lambda$-calculus 
maps to reduction in our translation. More specifically, if $M \Step M'$ then we must have 
that $\llbracket M \rrbracket_\Name{x} \Step \llbracket M' \rrbracket_\Name{x}$. \\

\noindent
We proceed by induction on the dynamics for the $\lambda$-calculus (\ref{fig: dr stlc}) and 
use cut reduction (\ref{fig: cr cp}) on each the translation until we cannot reduce terms any 
further. However, before we can begin our proof, just like in the dynamics for the $\lambda$-
calculus, we need the substitution lemma. \\

\subsection{Substitution}

\textbf{Substitution:} If we have some substitution $M [N/x]$ in the $\lambda$-calculus, then it's 
translation $\llbracket M [N/x] \rrbracket_\Name{y}$ is equivalent to the following in classical processes:
$\Proc{\New*{x}{}{\ofc \In{x}{z}{\llbracket N \rrbracket_\Name{z}} \mathbin{|} \llbracket M \rrbracket_\Name{y}}}$. \\

\noindent
We will now prove that this translation of substituition is correct by doing case analysis 
on the definition of substituition for the $\lambda$-calculus. 

\begin{mathpar}
  \begin{array}{rll}
    % x[N/x]
    \llbracket x [N/x] \rrbracket_\Name{y} &\EqDef &\llbracket N \rrbracket_\Name{y} \\
    \llbracket x [N/x] \rrbracket_\Name{y} &\EqDef &\Proc{\New*{x}{}{\ofc \In{x}{z}{\llbracket N \rrbracket_\Name{z}} \mathbin{|} \llbracket x \rrbracket_\Name{y}}} \\
    &= &\Proc{\New*{x}{}{\ofc \In{x}{z}{\llbracket N \rrbracket_\Name{z}} \mathbin{|} \whynot \Out{x}{z}{\Link{z}{y}}}} \\
    &\Longrightarrow_{\beta_{\ofc \whynot}} &\Proc{\New*{z}{}{\llbracket N \rrbracket_\Name{z} \mathbin{|} \Link{z}{y}}} \\
    &\Longrightarrow_{axCut} &\Proc{\llbracket N \rrbracket_\Name{z}[y/z]} \\
    &= &\llbracket N \rrbracket_\Name{y} \\\\
  \end{array}
\end{mathpar}

\begin{mathpar}
  \begin{array}{rll}
    % w[N/x]
    \llbracket w [N/x] \rrbracket_\Name{y} &\EqDef &\llbracket w \rrbracket_\Name{y} \\
    \llbracket w [N/x] \rrbracket_\Name{y} &\EqDef &\Proc{\New*{x}{}{\ofc \In{x}{z}{\llbracket N \rrbracket_\Name{z}} \mathbin{|} \llbracket w \rrbracket_\Name{y}}} \\
    &= &\Proc{\New*{x}{}{\ofc \In{x}{z}{\llbracket N \rrbracket_\Name{z}} \mathbin{|} \whynot \Out{w}{u}{\Link{u}{y}}}} \\
    &\Longrightarrow_{\beta{\ofc W}} &\Proc{\whynot \Out{w}{u}{\Link{u}{y}}} \\
    &= &\llbracket w \rrbracket_\Name{y} \\\\
  \end{array}
\end{mathpar}

\begin{mathpar}
  \begin{array}{rll}
    % <>[N/x]
    \llbracket \UnitV [N/x] \rrbracket_\Name{y} &\EqDef &\llbracket \UnitV \rrbracket_\Name{y} \\
    \llbracket \UnitV [N/x] \rrbracket_\Name{y} &\EqDef &\Proc{\New*{x}{}{\ofc \In{x}{z}{\llbracket N \rrbracket_\Name{z}} \mathbin{|} \llbracket \UnitV \rrbracket_\Name{y}}} \\
    &= &\Proc{\New*{x}{}{\ofc \In{x}{z}{\llbracket N \rrbracket_\Name{z}} \mathbin{|} \EmCase{y}}} \\
    &\equiv_{swap} &\Proc{\New*{x}{}{\EmCase{y} \mathbin{|} \ofc \In{x}{z}{\llbracket N \rrbracket_\Name{z}}}} \\
    &\Longrightarrow_{K \top} &\Proc{\EmCase{y}} \\
    &= &\llbracket \UnitV \rrbracket_\Name{y} \\
  \end{array}
\end{mathpar}

\begin{mathpar}
  \begin{array}{rll}
    % inl(M)[N/x]
    \llbracket \LamInl{M}[N/x] \rrbracket_\Name{y} &\EqDef &\llbracket \LamInl{M [N/x]} \rrbracket_\Name{y} \\
    \llbracket \LamInl{M}[N/x] \rrbracket_\Name{y} &\EqDef &\Proc{\New*{x}{}{\ofc \In{x}{z}{\llbracket N \rrbracket_\Name{z}} \mathbin{|} \llbracket \LamInl{M} \rrbracket_\Name{y}}} \\
    &= &\Proc{\New*{x}{}{\ofc \In{x}{z}{\llbracket N \rrbracket_\Name{z}} \mathbin{|} \Inl{y}{\ofc \In{y}{w}{\llbracket M \rrbracket_\Name{w}}}}} \\
    &\equiv_{swap} &\Proc{\New*{x}{}{\Inl{y}{\ofc \In{y}{w}{\llbracket M \rrbracket_\Name{w}}} \mathbin{|} \ofc \In{x}{z}{\llbracket N \rrbracket_\Name{z}}}} \\
    &\Longrightarrow_{K \oplus} &\Proc{\Inl{y}{\New*{x}{}{\ofc \In{y}{w}{\llbracket M \rrbracket_\Name{w}} \mathbin{|} \ofc \In{x}{z}{\llbracket N \rrbracket_\Name{z}}}}} \\
    &\Longrightarrow_{K \ofc} &\Proc{\Inl{y}{\ofc \In{y}{w}{\New*{x}{}{\llbracket M \rrbracket_\Name{w} \mathbin{|} \ofc \In{x}{z}{\llbracket N \rrbracket_\Name{z}}}}}} \\
    &\equiv_{swap} &\Proc{\Inl{y}{\ofc \In{y}{w}{\New*{x}{}{\ofc \In{x}{z}{\llbracket N \rrbracket_\Name{z}} \mathbin{|} \llbracket M \rrbracket_\Name{w}}}}} \\
    &= &\Proc{\Inl{y}{\ofc \In{y}{w}{\llbracket M [N/x] \rrbracket_\Name{w}}}} \\
    &= &\llbracket \LamInl{M [N/x]} \rrbracket_\Name{y} \\\\
  \end{array}
\end{mathpar}

\begin{mathpar}
  \begin{array}{rll}
    % case
    \llbracket \LamCase{M}{w}{P}{w}{Q} [N/x] \rrbracket_\Name{y} &\EqDef &\llbracket \LamCase{M [N/x]}{w}{P [N/x]}{w}{Q [N/x]} \rrbracket_\Name{y} \\
    \llbracket \LamCase{M}{w}{P}{w}{Q} [N/x] \rrbracket_\Name{y} &\EqDef &\Proc{\New*{x}{}{\ofc \In{x}{z}{\llbracket N \rrbracket_\Name{z}} \mathbin{|} \llbracket \LamCase{M}{w}{P}{w}{Q} \rrbracket_\Name{y}}} \\
    &= &\Proc{\New*{x}{}{\ofc \In{x}{z}{\llbracket N \rrbracket_\Name{z}} \mathbin{|} \New*{w}{}{\llbracket M \rrbracket_\Name{w} \mathbin{|} \Casep{w}{\llbracket P \rrbracket_\Name{y}}{\llbracket Q \rrbracket_\Name{y}}}}} \\
    &\equiv_{assoc} &\Proc{\New*{w}{}{\New*{x}{}{\ofc \In{x}{z}{\llbracket N \rrbracket_\Name{z}} \mathbin{|} \llbracket M \rrbracket_\Name{w}} \mathbin{|} \Casep{w}{\llbracket P \rrbracket_\Name{y}}{\llbracket Q \rrbracket_\Name{y}}}} \\
    &= &\Proc{\New*{w}{}{\New*{x}{}{\ofc \In{x}{z}{\llbracket N \rrbracket_\Name{z}} \mathbin{|} \llbracket M \rrbracket_\Name{w}[u/x][x/u]} \mathbin{|} \Casep{w}{\llbracket P \rrbracket_\Name{y}}{\llbracket Q \rrbracket_\Name{y}}}} \\
    &\Longrightarrow_{\beta_{\ofc C}} &\Proc{\New*{w}{}{\New*{x}{}{\ofc \In{x}{z}{\llbracket N \rrbracket_\Name{z}} \mathbin{|} \New*{u}{}{\ofc \In{u}{z}{\llbracket N \rrbracket_\Name{z}} \mathbin{|} \llbracket M \rrbracket_\Name{w}[u/x]}} \mathbin{|} \Casep{w}{\llbracket P \rrbracket_\Name{y}}{\llbracket Q \rrbracket_\Name{y}}}} \\
    &= &\Proc{\New*{w}{}{\New*{x}{}{\ofc \In{x}{z}{\llbracket N \rrbracket_\Name{z}} \mathbin{|} \llbracket (M[u/x])[N/u] \rrbracket_\Name{w}} \mathbin{|} \Casep{w}{\llbracket P \rrbracket_\Name{y}}{\llbracket Q \rrbracket_\Name{y}}}} \\
    &= &\Proc{\New*{w}{}{\New*{x}{}{\ofc \In{x}{z}{\llbracket N \rrbracket_\Name{z}} \mathbin{|} \llbracket M[N/x] \rrbracket_\Name{w}} \mathbin{|} \Casep{w}{\llbracket P \rrbracket_\Name{y}}{\llbracket Q \rrbracket_\Name{y}}}} \\
    &\equiv_{assoc} &\Proc{\New*{x}{}{\ofc \In{x}{z}{\llbracket N \rrbracket_\Name{z}} \mathbin{|} \New*{w}{}{\llbracket M[N/x] \rrbracket_\Name{w} \mathbin{|} \Casep{w}{\llbracket P \rrbracket_\Name{y}}{\llbracket Q \rrbracket_\Name{y}}}}} \\
    &\equiv_{swap} &\Proc{\New*{x}{}{\ofc \In{x}{z}{\llbracket N \rrbracket_\Name{z}} \mathbin{|} \New*{w}{}{\Casep{w}{\llbracket P \rrbracket_\Name{y}}{\llbracket Q \rrbracket_\Name{y}} \mathbin{|} \llbracket M[N/x] \rrbracket_\Name{w}}}} \\
    &\equiv_{assoc} &\Proc{\New*{w}{}{\New*{x}{}{\ofc \In{x}{z}{\llbracket N \rrbracket_\Name{z}} \mathbin{|} \Casep{w}{\llbracket P \rrbracket_\Name{y}}{\llbracket Q \rrbracket_\Name{y}}} \mathbin{|} \llbracket M[N/x] \rrbracket_\Name{w}}} \\
    &\equiv_{swap} &\Proc{\New*{w}{}{\New*{x}{}{\Casep{w}{\llbracket P \rrbracket_\Name{y}}{\llbracket Q \rrbracket_\Name{y}} \mathbin{|} \ofc \In{x}{z}{\llbracket N \rrbracket_\Name{z}}} \mathbin{|} \llbracket M[N/x] \rrbracket_\Name{w}}} \\
    &\Longrightarrow_{K \with} &\Proc{\New*{w}{}{\Casep{w}{\New*{x}{}{\ofc \In{x}{z}{\llbracket N \rrbracket_\Name{z}} \mathbin{|} \llbracket P \rrbracket_\Name{y}}}{\New*{x}{}{\ofc \In{x}{z}{\llbracket N \rrbracket_\Name{z}} \mathbin{|} \llbracket Q \rrbracket_\Name{y}}} \mathbin{|} \llbracket M[N/x] \rrbracket_\Name{w}}} \\
    &\equiv_{swap} &\Proc{\New*{w}{}{\Casep{w}{\New*{x}{}{\llbracket P \rrbracket_\Name{y} \mathbin{|} \ofc \In{x}{z}{\llbracket N \rrbracket_\Name{z}}}}{\New*{x}{}{\llbracket Q \rrbracket_\Name{y} \mathbin{|} \ofc \In{x}{z}{\llbracket N \rrbracket_\Name{z}}}} \mathbin{|} \llbracket M[N/x] \rrbracket_\Name{w}}} \\
    &= &\Proc{\New*{w}{}{\Casep{w}{\llbracket P [N/x] \rrbracket_\Name{y}}{\llbracket Q [N/x] \rrbracket_\Name{y}} \mathbin{|} \llbracket M [N/x] \rrbracket_\Name{w}}} \\
    &\equiv_{swap} &\Proc{\New*{w}{}{\llbracket M [N/x] \rrbracket_\Name{w} \mathbin{|} \Casep{w}{\llbracket P [N/x] \rrbracket_\Name{y}}{\llbracket Q [N/x] \rrbracket_\Name{y}}}} \\
    &= &\llbracket \LamCase{M [N/x]}{w}{P [N/x]}{w}{Q [N/x]} \rrbracket_\Name{y} \\
  \end{array}
\end{mathpar}

\begin{mathpar}
  \begin{array}{rll}
    % <P, Q>[N/x]
    \llbracket \Tuple{P}{Q} [N/x] \rrbracket_\Name{y} &\EqDef &\llbracket \Tuple{P [N/x]}{Q [N/x]} \rrbracket_\Name{y} \\
    \llbracket \Tuple{P}{Q} [N/x] \rrbracket_\Name{y} &\EqDef &\Proc{\New*{x}{}{\ofc \In{x}{z}{\llbracket N \rrbracket_\Name{z}} \mathbin{|} \llbracket \Tuple{P}{Q} \rrbracket_\Name{y}}} \\
    &= &\Proc{\New*{x}{}{\ofc \In{x}{z}{\llbracket N \rrbracket_\Name{z}} \mathbin{|} \Casep{y}{\llbracket P \rrbracket_\Name{y}}{\llbracket Q \rrbracket_\Name{y}}}} \\
    &\equiv_{swap} &\Proc{\New*{x}{}{\Casep{y}{\llbracket P \rrbracket_\Name{y}}{\llbracket Q \rrbracket_\Name{y}} \mathbin{|} \ofc \In{x}{z}{\llbracket N \rrbracket_\Name{z}}}} \\
    &\Longrightarrow_{K \with} & \Proc{\Casep{y}{\New*{x}{}{\llbracket P \rrbracket_\Name y \mathbin{|} \ofc \In{x}{z}{\llbracket N \rrbracket_\Name{z}}}}{\llbracket Q \rrbracket_\Name{y} \mathbin{|} \ofc \In{x}{z}{\llbracket N \rrbracket_\Name{z}}}} \\
    &\equiv_{swap} &\Proc{\Casep{y}{\New*{x}{}{\ofc \In{x}{z}{\llbracket N \rrbracket_\Name{z}} \mathbin{|} \llbracket P \rrbracket_\Name{y}}}{\New*{x}{}{\ofc \In{x}{z}{\llbracket N \rrbracket_\Name{z}} \mathbin{|} \llbracket Q \rrbracket_\Name{y}}}} \\
    &= &\Proc{\Casep{y}{\llbracket P [N/x] \rrbracket_\Name{y}}{\llbracket Q [N/x] \rrbracket_\Name{y}}} \\
    &= &\llbracket \Tuple{P [N/x]}{Q [N/x]} \rrbracket_\Name{y} \\
  \end{array}
\end{mathpar}

\begin{mathpar}
  \begin{array}{rll}
    % pi(M)[N/x]
    \llbracket \Proj{M}[N/x] \rrbracket_\Name{y} &\EqDef &\llbracket \Proj{M [N/x]} \rrbracket_\Name{y} \\
    \llbracket \Proj{M}[N/x] \rrbracket_\Name{y} &\EqDef &\Proc{\New*{x}{}{\ofc \In{x}{z}{\llbracket N \rrbracket_\Name{z}} \mathbin{|} \llbracket \Proj{M} \rrbracket_\Name{y}}} \\
    &= &\Proc{\New*{x}{}{\ofc \In{x}{z}{\llbracket N \rrbracket_\Name{z}} \mathbin{|} \New*{w}{}{\Inl{w}{\Link{w}{y}} \mathbin{|} \llbracket M \rrbracket_\Name{w}}}} \\
    &\equiv_{swap} &\Proc{\New*{x}{}{\ofc \In{x}{z}{\llbracket N \rrbracket_\Name{z}} \mathbin{|} \New*{w}{}{\llbracket M \rrbracket_\Name{w} \mathbin{|} \Inl{w}{\Link{w}{y}}}}} \\
    &\equiv_{assoc} &\Proc{\New*{w}{}{\New*{x}{}{\ofc \In{x}{z}{\llbracket N \rrbracket_\Name{z}} \mathbin{|} \llbracket M \rrbracket_\Name{w}} \mathbin{|} \Inl{w}{\Link{w}{y}}}} \\
    &\equiv_{swap} &\Proc{\New*{w}{}{\Inl{w}{\Link{w}{y}} \mathbin{|} \New*{x}{}{\ofc \In{x}{z}{\llbracket N \rrbracket_\Name{z}} \mathbin{|} \llbracket M \rrbracket_\Name{w}}}} \\
    &= &\Proc{\New*{w}{}{\Inl{w}{\Link{w}{y}} \mathbin{|} \llbracket M [N/x] \rrbracket_\Name{w}}} \\
    &= &\llbracket \Proj{M [N/x]} \rrbracket_\Name{y} \\
  \end{array}
\end{mathpar}

\begin{mathpar}
  \begin{array}{rll}
    % \w.M[N/x]
    \llbracket (\Lam{w}{M})[N/x] \rrbracket_\Name{y} &\EqDef &\llbracket \Lam{w}{M[N/x]} \rrbracket_\Name{y} \\
    \llbracket (\Lam{w}{M})[N/x] \rrbracket_\Name{y} &\EqDef &\Proc{\New*{x}{}{\ofc \In{x}{z}{\llbracket N \rrbracket_\Name{z}} \mathbin{|} \llbracket \Lam{w}{M} \rrbracket_\Name{y}}} \\
    &= &\Proc{\New*{x}{}{\ofc \In{x}{z}{\llbracket N \rrbracket_\Name{z}} \mathbin{|} \In{y}{w}{\llbracket M \rrbracket_\Name{y}}}} \\
    &\equiv_{swap} &\Proc{\New*{x}{}{\In{y}{w}{\llbracket M \rrbracket_\Name{y}} \mathbin{|} \ofc \In{x}{z}{\llbracket N \rrbracket_\Name{z}}}} \\
    &\Longrightarrow_{K \with} &\Proc{\In{y}{w}{\New*{x}{}{\llbracket M \rrbracket_\Name{y} \mathbin{|} \ofc \In{x}{z}{\llbracket N \rrbracket_\Name{z}}}}} \\
    &\equiv_{swap} &\Proc{\In{y}{w}{\New*{x}{}{\ofc \In{x}{z}{\llbracket N \rrbracket_\Name{z}} \mathbin{|} \llbracket M \rrbracket_\Name{y}}}} \\
    &= &\Proc{\In{y}{w}{\llbracket M [N/x] \rrbracket_\Name{y}}} \\
    &= &\llbracket \Lam{w}{M[N/x]} \rrbracket_\Name{y} \\
  \end{array}
\end{mathpar}

\begin{mathpar}
  \hspace{-3em}
  \begin{array}{rll}
    % P(Q)[N/x]
    \llbracket (P(Q)) [N/x] \rrbracket_\Name{y} &\EqDef &\llbracket (P [N/x](Q [N/x])) \rrbracket_\Name{y} \\
    \llbracket (P(Q)) [N/x] \rrbracket_\Name{y} &\EqDef &\Proc{\New*{x}{}{\ofc \In{x}{z}{\llbracket N \rrbracket_\Name{z}} \mathbin{|} \llbracket P(Q) \rrbracket_\Name{y}}} \\
    &= &\Proc{\New*{x}{}{\ofc \In{x}{z}{\llbracket N \rrbracket_\Name{z}} \mathbin{|} \New*{\alpha}{}{\New*{\sigma}{}{\Out{\sigma}{w}{(\Link{w}{\alpha} \mathbin{|} \Link{y}{\sigma})} \mathbin{|} \llbracket P \rrbracket_\Name{\sigma}} \mathbin{|} \ofc \In{\alpha}{u}{\llbracket Q \rrbracket_\Name{u}}}}} \\
    &\equiv_{assoc} &\Proc{\New*{\alpha}{}{\New*{x}{}{\ofc \In{x}{z}{\llbracket N \rrbracket_\Name{z}} \mathbin{|} \New*{\sigma}{}{\Out{\sigma}{w}{(\Link{w}{\alpha} \mathbin{|} \Link{y}{\sigma})} \mathbin{|} \llbracket P \rrbracket_\Name{\sigma}}} \mathbin{|} \ofc \In{\alpha}{u}{\llbracket Q \rrbracket_\Name{u}}}} \\
    &= &\Proc{\New*{\alpha}{}{\New*{x}{}{\ofc \In{x}{z}{\llbracket N \rrbracket_\Name{z}} \mathbin{|} \New*{\sigma}{}{\Out{\sigma}{w}{(\Link{w}{\alpha} \mathbin{|} \Link{y}{\sigma})} \mathbin{|} \llbracket P \rrbracket_\Name{\sigma}}[\tau /x][x/ \tau]} \mathbin{|} \ofc \In{\alpha}{u}{\llbracket Q \rrbracket_\Name{u}}}} \\
    &\Longrightarrow_{\beta_{\ofc C}} &\Proc{\New*{\alpha}{}{\New*{x}{}{\ofc \In{x}{z}{\llbracket N \rrbracket_\Name{z}} \mathbin{|} \New*{\tau}{}{\ofc \In{\tau}{z}{\llbracket N \rrbracket_\Name{z}} \mathbin{|} \New*{\sigma}{}{\Out{\sigma}{w}{(\Link{w}{\alpha} \mathbin{|} \Link{y}{\sigma})} \mathbin{|} \llbracket P \rrbracket_\Name{\sigma}}[\tau /x]}} \mathbin{|} \ofc \In{\alpha}{u}{\llbracket Q \rrbracket_\Name{u}}}} \\
    &= &\Proc{\New*{\alpha}{}{\New*{x}{}{\ofc \In{x}{z}{\llbracket N \rrbracket_\Name{z}} \mathbin{|} \New*{\tau}{}{\ofc \In{\tau}{z}{\llbracket N \rrbracket_\Name{z}} \mathbin{|} \New*{\sigma}{}{\Out{\sigma}{w}{(\Link{w}{\alpha} \mathbin{|} \Link{y}{\sigma})[\tau /x]} \mathbin{|} \llbracket P \rrbracket_\Name{\sigma}[\tau /x]}}} \mathbin{|} \ofc \In{\alpha}{u}{\llbracket Q \rrbracket_\Name{u}}}} \\
    &\equiv_{swap} &\Proc{\New*{\alpha}{}{\New*{x}{}{\ofc \In{x}{z}{\llbracket N \rrbracket_\Name{z}} \mathbin{|} \New*{\tau}{}{\ofc \In{\tau}{z}{\llbracket N \rrbracket_\Name{z}} \mathbin{|} \New*{\sigma}{}{\llbracket P \rrbracket_\Name{\sigma}[\tau /x] \mathbin{|} \Out{\sigma}{w}{(\Link{w}{\alpha} \mathbin{|} \Link{y}{\sigma})}}}} \mathbin{|} \ofc \In{\alpha}{u}{\llbracket Q \rrbracket_\Name{u}}}} \\
    &\equiv_{assoc} &\Proc{\New*{\alpha}{}{\New*{x}{}{\ofc \In{x}{z}{\llbracket N \rrbracket_\Name{z}} \mathbin{|} \New*{\sigma}{}{\New*{\tau}{}{\ofc \In{\tau}{z}{\llbracket N \rrbracket_\Name{z}} \mathbin{|} \llbracket P \rrbracket_\Name{\sigma}[\tau /x]} \mathbin{|} \Out{\sigma}{w}{(\Link{w}{\alpha} \mathbin{|} \Link{y}{\sigma})}} \mathbin{|} \ofc \In{\alpha}{u}{\llbracket Q \rrbracket_\Name{u}}}}} \\
    &= &\Proc{\New*{\alpha}{}{\New*{x}{}{\ofc \In{x}{z}{\llbracket N \rrbracket_\Name{z}} \mathbin{|} \New*{\sigma}{}{\llbracket (P [\tau /x])[N/ \tau] \rrbracket_\Name{\sigma} \mathbin{|} \Out{\sigma}{w}{(\Link{w}{\alpha} \mathbin{|} \Link{y}{\sigma})}} \mathbin{|} \ofc \In{\alpha}{u}{\llbracket Q \rrbracket_\Name{u}}}}} \\
    &= &\Proc{\New*{\alpha}{}{\New*{x}{}{\ofc \In{x}{z}{\llbracket N \rrbracket_\Name{z}} \mathbin{|} \New*{\sigma}{}{\llbracket P [N/x] \rrbracket_\Name{\sigma} \mathbin{|} \Out{\sigma}{w}{(\Link{w}{\alpha} \mathbin{|} \Link{y}{\sigma})}} \mathbin{|} \ofc \In{\alpha}{u}{\llbracket Q \rrbracket_\Name{u}}}}} \\
    &\equiv_{assoc} &\Proc{\New*{x}{}{\ofc \In{x}{z}{\llbracket N \rrbracket_\Name{z}} \mathbin{|} \New*{\alpha}{}{\New*{\sigma}{}{\llbracket P [N/x] \rrbracket_\Name{\sigma} \mathbin{|} \Out{\sigma}{w}{(\Link{w}{\alpha} \mathbin{|} \Link{y}{\sigma})}} \mathbin{|} \ofc \In{\alpha}{u}{\llbracket Q \rrbracket_\Name{u}}}}} \\
    &\equiv_{swap} &\Proc{\New*{x}{}{\ofc \In{x}{z}{\llbracket N \rrbracket_\Name{z}} \mathbin{|} \New*{\alpha}{}{\ofc \In{\alpha}{u}{\llbracket Q \rrbracket_\Name{u}} \mathbin{|} \New*{\sigma}{}{\llbracket P [N/x] \rrbracket_\Name{\sigma} \mathbin{|} \Out{\sigma}{w}{(\Link{w}{\alpha} \mathbin{|} \Link{y}{\sigma})}}}}} \\
    &\equiv_{assoc} &\Proc{\New*{\alpha}{}{\New*{x}{}{\ofc \In{x}{z}{\llbracket N \rrbracket_\Name{z}} \mathbin{|} \ofc \In{\alpha}{u}{\llbracket Q \rrbracket_\Name{u}}} \mathbin{|} \New*{\sigma}{}{\llbracket P [N/x] \rrbracket_\Name{\sigma} \mathbin{|} \Out{\sigma}{w}{(\Link{w}{\alpha} \mathbin{|} \Link{y}{\sigma})}}}} \\
    &\Longrightarrow_{K \ofc} &\Proc{\New*{\alpha}{}{\ofc \In{\alpha}{u}{\New*{x}{}{\ofc \In{x}{z}{\llbracket N \rrbracket_\Name{z}} \mathbin{|} \llbracket Q \rrbracket_\Name{u}}} \mathbin{|} \New*{\sigma}{}{\llbracket P [N/x] \rrbracket_\Name{\sigma} \mathbin{|} \Out{\sigma}{w}{(\Link{w}{\alpha} \mathbin{|} \Link{y}{\sigma})}}}} \\
    &= &\Proc{\New*{\alpha}{}{\ofc \In{\alpha}{u}{\llbracket Q [N/x] \rrbracket_\Name{u}} \mathbin{|} \New*{\sigma}{}{\llbracket P [N/x] \rrbracket_\Name{\sigma} \mathbin{|} \Out{\sigma}{w}{(\Link{w}{\alpha} \mathbin{|} \Link{y}{\sigma})}}}} \\
    &\equiv_{swap} &\Proc{\New*{\alpha}{}{\New*{\sigma}{}{\llbracket P [N/x] \rrbracket_\Name{\sigma} \mathbin{|} \Out{\sigma}{w}{(\Link{w}{\alpha} \mathbin{|} \Link{y}{\sigma})}} \mathbin{|} \ofc \In{\alpha}{u}{\llbracket Q [N/x] \rrbracket_\Name{u}}}} \\
    &\equiv_{swap} &\Proc{\New*{\alpha}{}{\New*{\sigma}{}{\Out{\sigma}{w}{(\Link{w}{\alpha} \mathbin{|} \Link{y}{\sigma})} \mathbin{|} \llbracket P [N/x] \rrbracket_\Name{\sigma}} \mathbin{|} \ofc \In{\alpha}{u}{\llbracket Q [N/x] \rrbracket_\Name{u}}}} \\
    &= &\llbracket (P [N/x](Q [N/x])) \rrbracket_\Name{y} \\
  \end{array}
\end{mathpar}

\subsection{Reduction Simulation}

Now we have proved that we have a valid substitution lemma, we can begin our proof of 
simulation to show that our translation is indeed correct.

\begin{mathpar}
  \begin{array}{rll}
    %case
    \Biggl\llbracket
    \inferH{S-Case-In-l}{
    }{
      \LamCase{\LamInl{M}}{y}{P}{y}{Q} \Step P[M/y]
    }\Biggr\rrbracket_\Name{x}
    &\EqDef
    &\inferH{}{
    }{
      \llbracket \LamCase{\LamInl{M}}{y}{P}{y}{Q} \rrbracket_\Name{x} \Step \llbracket P[M/y] \rrbracket_\Name{x}
    } \\\\
    \llbracket \LamCase{\LamInl{M}}{y}{P}{y}{Q} \rrbracket_\Name{x}
    &\EqDef
    &\Proc{\New*{y}{}{\llbracket \LamInl{M} \rrbracket_\Name{y} \mathbin{|} \Casep{y}{\llbracket P \rrbracket_\Name{x}}{\llbracket Q \rrbracket_\Name{x}}}} \\\\
    &= &\Proc{\New*{y}{}{\Inl{y}{\ofc \In{y}{z}{\llbracket M \rrbracket_\Name{z}} \mathbin{|} \Casep{y}{\llbracket P \rrbracket_\Name{x}}{\llbracket Q \rrbracket_\Name{x}}}}} \\\\
    &\Longrightarrow_{\beta_{\oplus \with}} &\Proc{\New*{y}{}{\ofc \In{y}{z}{\llbracket M \rrbracket_\Name{z}} \mathbin{|} \llbracket P \rrbracket_\Name{x}}} \\\\      
    &= &\llbracket P[M/y] \rrbracket_\Name{x} \\\\
  \end{array}
\end{mathpar}

\noindent
For \ruleref{S-Case-In-l} we simply extract the translation of the case term then use the 
$\beta_{\oplus \with}$ cut reduction rule to handle the 
$\Proc{\New*{y}{}{\Inl{y}{} \mathbin{|} \Casep{y}{}{}}}$ and achieve the translation of 
the result. The proof for case-in-r is analogous. \\

\begin{mathpar}
  \begin{array}{rll}
    \Biggl\llbracket
    \inferH{S-Case}{
      M \Step M'
    }{
      \LamCase{M}{y}{P}{y}{Q} \Step \LamCase{M'}{y}{P}{y}{Q}
    }\Biggr\rrbracket_\Name{x}
    &\EqDef 
    &\inferH{}{
      \llbracket M \rrbracket_\Name{y} \Step \llbracket M' \rrbracket_\Name{y}
    }{
      \llbracket \LamCase{M}{y}{P}{y}{Q} \rrbracket_\Name{x} \Step \llbracket \LamCase{M'}{y}{P}{y}{Q} \rrbracket_\Name{x}
    } \\\\
    \llbracket \LamCase{M}{y}{P}{y}{Q} \rrbracket_\Name{x} 
    &\EqDef 
    &\Proc{\New*{y}{}{\llbracket M \rrbracket_\Name{y} \mathbin{|} \Casep{y}{\llbracket P \rrbracket_\Name{x}}{\llbracket Q \rrbracket_\Name{x}}}} \\\\
    &\Longrightarrow_{SCase} &\Proc{\New*{y}{}{\llbracket M' \rrbracket_\Name{y} \mathbin{|} \Casep{y}{\llbracket P \rrbracket_\Name{x}}{\llbracket Q \rrbracket_\Name{x}}}} \\\\
    &= &\llbracket \LamCase{M'}{y}{P}{y}{Q} \rrbracket_\Name{x} \\\\
  \end{array}
\end{mathpar}

\noindent
We know that $\llbracket M \rrbracket_\Name{y} \Step \llbracket M' \rrbracket_\Name{y}$ 
by our induction hypothesis, so applying the \ruleref{S-Case} rule we obtain the required result. \\

\begin{mathpar}
  \begin{array}{rll}
    %proj
    \Biggl\llbracket
    \inferH{S-Prj-Pair-l}{
    }{
      \Proj{\Tuple{M}{N}} \Step M
    }\Biggr\rrbracket_\Name{x}
    &\EqDef
    &\inferH{}{
    }{
      \llbracket \Proj{\Tuple{M}{N}} \rrbracket_\Name{x} \Step \llbracket M \rrbracket_\Name{x}
    } \\\\
    \llbracket \Proj{\Tuple{M}{N}} \rrbracket_\Name{x} 
    &\EqDef
    &\Proc{\New*{y}{}{\Inl{y}{\Link{y}{x}} \mathbin{|} \llbracket \Tuple{M}{N} \rrbracket_\Name{y}}} \\\\
    &= &\Proc{\New*{y}{}{\Inl{y}{\Link{y}{x}} \mathbin{|} \Casep{y}{\llbracket M \rrbracket_\Name{y}}{\llbracket N \rrbracket_\Name{y}}}} \\\\
    &\Longrightarrow_{\beta_{\oplus \with}} &\Proc{\New*{y}{}{\Link{y}{x} \mathbin{|} \llbracket M \rrbracket_\Name{y}}} \\\\
    &\Longrightarrow_{axCut} &\Proc{\llbracket M \rrbracket_\Name{y}[\Name{x} / \Name{y}]} \\\\
    &= &\llbracket M \rrbracket_\Name{x} \\\\
  \end{array}
\end{mathpar}

\noindent
For \ruleref{S-Prj-Pair-l} we find the translation of the term then use the $\beta_{\oplus \with}$ 
cut reduction rule again to handle the $\Proc{\New*{y}{}{\Inl{y}{} \mathbin{|} \Casep{y}{}{}}}$, 
then we use the \ruleref{AxCut} rule to substitute $\Name{x}$ for $\Name{y}$ in $\llbracket M \rrbracket_\Name{y}$. 
This results in the translation of $M$ with respect to $\Name{x}$ which is what we needed. Again, the 
proof for prj-pair-r will be much the same. \\

\begin{mathpar}
  \begin{array}{rll}
    \Biggl\llbracket
    \inferH{S-Prj-l}{
      M \Step M'
    }{
      \Proj{M} \Step \Proj{M'}
    }\Biggr\rrbracket_\Name{x}
    &\EqDef
    &\inferH{}{
      \llbracket M \rrbracket_\Name{y} \Step \llbracket M' \rrbracket_\Name{y}
    }{
      \llbracket \Proj{M} \rrbracket_\Name{x} \Step \llbracket \Proj{M'} \rrbracket_\Name{x}
    } \\\\
    \llbracket \Proj{M} \rrbracket_\Name{x}
    &\EqDef 
    &\Proc{\New*{y}{}{\Inl{y}{\Link{y}{x}} \mathbin{|} {\llbracket M \rrbracket}_\Name{y}}} \\\\
    &\Longrightarrow_{SPrjL} &\Proc{\New*{y}{}{\Inl{y}{\Link{y}{x}} \mathbin{|} {\llbracket M' \rrbracket}_\Name{y}}} \\\\
    &= &\llbracket \Proj{M'} \rrbracket_\Name{x} \\\\
  \end{array}
\end{mathpar}

\noindent
We know by our induction hypothesis that $\llbracket M \rrbracket_\Name{y} \Step \llbracket M' \rrbracket_\Name{y}$, 
so applying the \ruleref{S-Prj-l} rule, we achieve the necessary result. The proof for S-Prj-r is analogous. \\

\begin{mathpar}
  \begin{array}{rll}
    %beta
    \Biggl\llbracket
    \inferH{S-Beta}{
    }{
      (\Lam{x}{M})(N) \Step M[N/x]
    }\Biggr\rrbracket_\Name{y}
    &\EqDef
    &\inferH{}{
    }{
      \llbracket (\Lam{x}{M})(N) \rrbracket_\Name{y} \Step \llbracket M[N/x] \rrbracket_\Name{y}
    } \\\\
    \llbracket (\Lam{x}{M})(N) \rrbracket_\Name{y} 
    &\EqDef
    &\Proc{\New*{x}{}{\Depar{x}{y}{\llbracket \lambda x.M \rrbracket_\Name{z}} \mathbin{|} \ofc \In{x}{u}{\llbracket N \rrbracket_\Name{u}}}} \\\\
    &= &\Proc{\New*{x}{}{\New*{z}{}{\Out{z}{w}{(\Link{w}{x} \mathbin{|} \Link{y}{z})} \mathbin{|} (\In{z}{x}{\llbracket M \rrbracket_\Name{z}})} \mathbin{|} \ofc \In{x}{u}{\llbracket N \rrbracket_\Name{u}}}} \\\\
    &= &\Proc{\New*{x}{}{\New*{z}{}{\Out{z}{w}{(\Link{w}{x} \mathbin{|} \Link{y}{z})} \mathbin{|} (\In{z}{x}{\llbracket M \rrbracket_\Name{z}})[w/x]} \mathbin{|} \ofc \In{x}{u}{\llbracket N \rrbracket_\Name{u}}}} \\\\
    &= &\Proc{\New*{x}{}{\New*{z}{}{\Out{z}{w}{(\Link{w}{x} \mathbin{|} \Link{y}{z})} \mathbin{|} (\In{z}{w}{\llbracket M \rrbracket_\Name{z}[w/x]})} \mathbin{|} \ofc \In{x}{u}{\llbracket N \rrbracket_\Name{u}}}} \\\\
    &\Longrightarrow_{\beta_{\otimes \parr}} &\Proc{\New*{x}{}{\New*{w}{}{\Link{w}{x} \mathbin{|} \New*{z}{}{\Link{y}{z} \mathbin{|} \llbracket M \rrbracket_\Name{z}}} \mathbin{|} \ofc \In{x}{u}{\llbracket N \rrbracket_\Name{u}}}} \\\\
    &\Longrightarrow_{axCut} &\Proc{\New*{x}{}{\New*{w}{}{\Link{w}{x} \mathbin{|} \llbracket M \rrbracket_\Name{y}} \mathbin{|} \ofc \In{x}{u}{\llbracket N \rrbracket_\Name{u}}}} \\\\
    &\Longrightarrow_{axCut} &\Proc{\New*{x}{}{\llbracket M \rrbracket_\Name{y} \mathbin{|} \ofc \In{x}{u}{\llbracket N \rrbracket_\Name{u}}}} \\\\
    &\equiv_{swap} &\Proc{\New*{x}{}{\ofc \In{x}{u}{\llbracket N \rrbracket_\Name{u}} \mathbin{|} \llbracket M \rrbracket_\Name{y}}} \\\\
    &= &\llbracket M[N/x] \rrbracket_\Name{y} \\\\
  \end{array}
\end{mathpar}

\noindent
For \ruleref{S-Beta} we begin as usual by extracting the translation of the term. Then we substitute 
$w$ for $x$ in $\Proc{\In{z}{w}{\llbracket M \rrbracket_\Name{z}}}$ so we can use the $\beta_{\otimes \parr}$ 
cut reduction rule to handle $\Proc{\New*{z}{}{\Out{z}{w}{} \mathbin{|} \In{z}{w}{}}}$. Then we use the axCut 
rule to substitute $y$ for $z$ in $\llbracket M \rrbracket_\Name{z}$ resulting in $\llbracket M \rrbracket_\Name{y}$. 
We then use axCut again to substitue $x$ for $w$ in $\llbracket M \rrbracket_\Name{y}$ which changes nothing. 
Finally we use the swap equivalence to switch the positions of the subterms in $\Proc{\New*{x}{}{\dots}}$ to achieve 
the translation of the result as required. \\

% -----------------------------------------------------------------------------
\chapter{Critical Evaluation}
\label{chap:evaluation}

{\bf A topic-specific chapter, of roughly $15$ pages} 
\vspace{1cm} 

\noindent
This chapter is intended to evaluate what you did.  The content is highly 
topic-specific, but for many projects will have flavours of the following:

\begin{enumerate}
\item functional  testing, including analysis and explanation of failure 
      cases,
\item behavioural testing, often including analysis of any results that 
      draw some form of conclusion wrt. the aims and objectives,
      and
\item evaluation of options and decisions within the project, and/or a
      comparison with alternatives.
\end{enumerate}

\noindent
This chapter often acts to differentiate project quality: even if the work
completed is of a high technical quality, critical yet objective evaluation 
and comparison of the outcomes is crucial.  In essence, the reader wants to
learn something, so the worst examples amount to simple statements of fact 
(e.g., ``graph X shows the result is Y''); the best examples are analytical 
and exploratory (e.g., ``graph X shows the result is Y, which means Z; this 
contradicts [1], which may be because I use a different assumption'').  As 
such, both positive {\em and} negative outcomes are valid {\em if} presented 
in a suitable manner.

% -----------------------------------------------------------------------------
\chapter{Conclusion}
\label{chap:conclusion}

{\bf A compulsory chapter,     of roughly $5$ pages} 
\vspace{1cm} 

% \noindent
% The concluding chapter of a dissertation is often underutilised because it 
% is too often left too close to the deadline: it is important to allocation
% enough attention.  Ideally, the chapter will consist of three parts:

% \begin{enumerate}
% \item (Re)summarise the main contributions and achievements, in essence
%       summing up the content.
% \item Clearly state the current project status (e.g., ``X is working, Y 
%       is not'') and evaluate what has been achieved with respect to the 
%       initial aims and objectives (e.g., ``I completed aim X outlined 
%       previously, the evidence for this is within Chapter Y'').  There 
%       is no problem including aims which were not completed, but it is 
%       important to evaluate and/or justify why this is the case.
% \item Outline any open problems or future plans.  Rather than treat this
%       only as an exercise in what you {\em could} have done given more 
%       time, try to focus on any unexplored options or interesting outcomes
%       (e.g., ``my experiment for X gave counter-intuitive results, this 
%       could be because Y and would form an interesting area for further 
%       study'' or ``users found feature Z of my software difficult to use,
%       which is obvious in hindsight but not during at design stage; to 
%       resolve this, I could clearly apply the technique of Smith [7]'').
% \end{enumerate}

% -----------------------------------------------------------------------------
% =============================================================================

% Finally, after the main matter, the back matter is specified.  This is
% typically populated with just the bibliography.  LaTeX deals with these
% in one of two ways, namely
%
% - inline, which roughly means the author specifies entries using the 
%   \bibitem macro and typesets them manually, or
% - using BiBTeX, which means entries are contained in a separate file
%   (which is essentially a databased) then inported; this is the 
%   approach used below, with the databased being dissertation.bib.
%
% Either way, the each entry has a key (or identifier) which can be used
% in the main matter to cite it, e.g., \cite{X}, \cite[Chapter 2}{Y}.

\backmatter

\printbibliography

% The dissertation concludes with a set of (optional) appendicies; these are 
% the same as chapters in a sense, but once signaled as being appendicies via
% the associated macro, LaTeX manages them appropriatly.

\appendix

\chapter{An Example Appendix}
\label{appx:example}

Content which is not central to, but may enhance the dissertation can be 
included in one or more appendices; examples include, but are not limited
to

\begin{itemize}
\item lengthy mathematical proofs, numerical or graphical results which 
      are summarised in the main body,
\item sample or example calculations, 
      and
\item results of user studies or questionnaires.
\end{itemize}

\noindent
Note that in line with most research conferences, the marking panel is not
obliged to read such appendices.

% -----------------------------------------------------------------------------
% =============================================================================

\end{document}
